Many real-world graphs that model complex systems 
exhibit an organization into subgraphs that are more 
densely connected on the inside than between each other. 
These subgraphs are called communities or modules or clusters 
and have no widely accepted definition~\cite{LF09, PDFV05}. 

Discovering the community structure of networks is an 
important problem in network science and is the subject 
of intensive research~\cite{GN02, GN04, CNM04, RCC04, DM04, PDFV05, NL07, 
BGLL08, RB08, RN09}. Existing community detection algorithms are 
classified by whether they find disjoint (non-overlapping) 
communitities or overlapping communitities. Typically finding 
overlapping communitities is a much harder problem and most of the 
earlier community detection algorithms focussed on finding disjoint 
communities. A comparative analysis of several community detection algorithms 
(both non-overlapping and overlapping) was presented by Lancichinetti and Fortunato 
in~\cite{LF09}. Their paper proposes a test framework 
which comprises of testing algorithms on a set of benchmark graphs 
known as the LFR-benchmark with the test results being evaluated against 
the ground truth using an information-theoretic measure known as the \emph{normalized 
mutual information}. We closely follow this test framework in this paper. 
In what follows, we briefly mention those algorithms that we find are interesting 
or that were reported to have done particularly well in~\cite{LF09}.

\paragraph{The Girvan-Newman algorithm.} 
One of the very first algorithms for detecting disjoint communities 
was due to Girvan and Newman~\cite{GN02, GN04}. Their 
algorithm starts with a network and iteratively removes edges based 
on a metric called \emph{edge betweenness}. The betweenness of an 
edge is defined as the number of shortest paths between vertex pairs 
that pass through that edge. After an edge is removed, betweenness 
scores are recalculated and an edge with the highest score is deleted. 
This procedure ends when the modularity of the resulting partition
reaches a maximum. Modularity is a measure that estimates the goodness 
of a partition by comparing the network with a so-called ``null model''
in which edges are added at random between the nodes of the network 
so as to preserve the degree sequence of the nodes. 

Formally, the \emph{modularity} of a partition is defined as:
\begin{equation}\label{eqn:modularity}
	Q = \frac{1}{2m} \sum_{i, j} \left ( A_{i j} - \frac{d_i d_j}{2m} \right ) \delta(i, j),
\end{equation}
where $A_{ij}$ represent the entries of the adjacency matrix of the network; $d_i$ is the 
degree of node $i$; $m$ is the number of edges in the network; and $\delta(i, j) = 1$ if nodes
$i$ and $j$ belong to the same set of the partition and $0$ otherwise. The term $d_i d_j / 2m$ 
represents the expected number of edges between nodes $i$ and $j$ if we consider a random model
in which each node $i$ has $d_i$ ``stubs'' and we are allowed to connect stubs at random to form edges. 
This is the null model against which the within-community edges of the partition is compared against.
The worst-case complexity of the Newman-Girvan algorithm is dominated by the time taken 
to compute the betweeness scores and is $O(m^2 n)$ for general graphs and $O(n^3)$ for sparse graphs.

\paragraph{The greedy algorithm for modularity optimization by Clauset et al.~\cite{CNM04}.}
This algorithm starts with each node being the sole member of a community of one, and 
repeatedly joins two communities whose amalgamation produces the largest increase in modularity. 
The algorithm makes use of efficient data structures and has a running time of $O(m \log^2 n)$, 
which for sparse graphs works out to $O(n \log^2 n)$. 

\paragraph{Fast Modularity Optimization by Blondel et al.} 
The algorithm of Blondel et al.~\cite{BGLL08} is 


