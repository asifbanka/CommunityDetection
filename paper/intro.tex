Many real-world graphs that model complex systems exhibit an organization 
into subgraphs, or \textit{communities} that are more densely connected on the inside than between each other. 
Social networks such as Facebook and LinkedIn divide into groups of friends 
or coworkers, or business partners; scientific collaboration networks divide 
themselves into research communities; the World Wide Web divides into groups 
of related webpages. The nature and number of communities provide 
a useful insight into the structure and organization of networks. 

Discovering the community structure of networks is an 
important problem in network science and is the subject 
of intensive research~\cite{GN02, GN04, CNM04, RCC04, DM04, PDFV05, NL07, 
BGLL08, RB08, RN09}. Existing community detection algorithms are 
distinguished by whether they find disjoint (non-overlapping) 
communities or overlapping communities. Typically finding 
overlapping communities is a much harder problem and most of the 
earlier community detection algorithms focused on finding disjoint 
communities. A comparative analysis of several community detection algorithms 
(both non-overlapping and overlapping) was presented by Lancichinetti and Fortunato 
in~\cite{LF09}. 
In this paper we closely follow their test framework, also called the LFR-benchmark.

The notion of a community is a loose one and there is no well-accepted definition.
A typical definition is to consider a partition of the node set of the network 
such that the density of inter-community edges is greater than the density of edges 
across communities. Different community detection algorithms formalize this
idea differently. One of the very first algorithms by
Girvan and Newman~\cite{GN02} introduced a measure known as \textit{modularity}
which, given a partition of the nodes of the network, compares the fraction of 
inter-community edges with the edges that would be present if the edges had been 
rewired randomly while preserving the node degrees. Other authors such as Palla 
\etal~\cite{PDFV05} declare communities as node sets that formed 
by overlapping maximal cliques. Rosvall and Bergstrom~\cite{RB08} 
define the goodness of a partition in terms of the number of bits required to 
describe per step of an infinite random walk in the network, the intuition being 
that in a ``correct'' partition, a random walker is likely to spend more time 
within communities rather than between communities, thereby decreasing the 
description of the walk.  

A severe restriction of many existing community detection algorithms 
is that they are too slow. Algorithms that optimize modularity typically 
take $O(n^2)$, even on sparse networks. The overlapping clique finding 
algorithm of Palla \etal~\cite{PDFV05} take exponential time in the worst case.
In other cases, derivation of worst-case running time bounds are ignored. 

\paragraph{This paper.}
Given that it is unlikely that users of community detection algorithms 
would unanimously settle on one definition of what constitutes a community, 
we feel that existing algorithms ignore the \emph{user perspective}.
To this end, we chose to design an algorithm that takes the network structure, 
as well as user preferences into account. 
The user is expected to classify a small set of nodes of the network 
into communities (which may be 6--8\% of the nodes of each community).
Obviously this is possible only when the user has some 
information about the network, such as, what it represents, which nodes 
are the important ones and into which communities they are classified. 
Such situations are actually quite likely. The user might have data only 
on the leading scientific authors in a co-authorship network 
and would like to find out the research areas of the remaining members of the network. 
He may either be interested in a broad partition of the network into into its main fields
or a fine grained decomposition into various subfields.
By labeling the known authors accordingly,
the user can specify which kind of partition he is interested in.
Another example would be the detection of trends in a social network.
Consider the case where one knows the political affiliations of some people 
and aims to discover political spectrum of the whole network, 
for example, to predict the outcome of an election. 


Another scenario where this may be applicable is in recommendation systems. 
One might know the preferences of some of the users of an online retail 
merchant possibly because they purchase items much more frequently than others. 
One could then use this in the network whose nodes consist of users, with two 
nodes connected by an edge if they represent users that had purchased similar products in the past. 
The idea now would be to use the knowledge of the prefererences of a few to 
predict the preferences of everyone in the network. 
\texttt{(jan: this recommender-system-example may not straight forward enough for the introduction?)}


An important characteristic of algorithms surveyed in~\cite{LF09} 
is that the algorithms either find disjoint communities or overlapping 
ones. Most algorithms solve the easier problem of finding disjoint communities. 
The ones that are designed to find overlapping communities such as the overlapping clique finding 
algorithm of Palla \etal~\cite{PDFV05} do not seem to yield very good results (see~\cite{LF09}).
Our algorithm naturally extends to the overlapping case. Of course, there is a higher 
price that has to be paid in that the number of nodes that need to be classified by the user 
typically is larger (5\% to 10\% of the nodes per community). The algorithm, however,
does not need any major changes and we view this is as an aesthetically pleasing 
feature of our approach. 

Thirdly, in many other approaches, the worst-case running time of the algorithms 
is neither stated nor analyzed. We show that our algorithm runs in time $O(k \cdot m \cdot \log n)$, 
where $k$ is the number of communities to be discovered (which is supplied by the user), 
$n$ and $m$ are the number of nodes and edges in the network. In the case of sparse graphs 
and a constant number of communities, the running time is $O(n \cdot \log n)$. Given that even 
an $O(n^2)$ time algorithm is too computationally expensive on many real world graphs, 
a nearly linear time algorithm often is the only feasible solution.   

Finally, we provide an extensive experimental evaluation of our algorithm on the LFR benchmark
introduced by Lancichinetti, Fortunato and Radicchi in~\cite{LFR08, LF09}. In order to ensure 
a fair comparison with other algorithms reviewed in~\cite{LF09}, 
we choose all parameters of the benchmark as in the original paper.


This paper is organized as follows. In Section~\ref{sec:major_algorithms}, we review some 
of the more influential algorithms in community detection. In Section~\ref{sec:algorithm}, we 
describe our algorithm and analyze its running time. In Sections~\ref{sec:experiment_setup} 
and~\ref{sec:experiment_results}, we present our experimental results. Finally we conclude 
in Section~\ref{sec:conclusions} with possibilities of how our approach might be extended. 



