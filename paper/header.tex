\usepackage[T1]{fontenc}
\usepackage{pgf}
\usepackage{tikz}
\usetikzlibrary{arrows,automata,positioning,calc}
\tikzset{
 mainNode/.style =
    { circle
    , draw
    %, fill=blue!20,
    %, font=\sffamily\Large\bfseries
    }
}
\usepackage[latin1]{inputenc}

\usepackage{float}

% FONTS and SYMBOLS
\usepackage{mathrsfs}
\usepackage{lmodern}
%\usepackage{txfonts}
\usepackage{paralist}

% MATHEMATICS
\usepackage{amsmath,amssymb,amsthm} % AMS packages
%\usepackage[warning, all]{onlyamsmath}

% COLORS and GRAPHICS
\usepackage{graphicx}
\usepackage[left=3cm,right=3cm,top=4cm,bottom=4cm]{geometry}

% FIGURES and TABLES
\usepackage{booktabs} % Better tables

% SUBFIGURES
\usepackage[small]{caption}
\usepackage{subcaption}

\usepackage{enumerate}
\usepackage{xspace}

\usepackage{microtype} % Prettifies pdf output. Uncomment if you have trouble with this. Will also reduce page count.

%\usepackage[ruled,longend,vlined]{algorithm2e}

\newcommand{\N}{\ensuremath{\mathbf{N}}\xspace} % Natural numbers
\newcommand{\R}{\ensuremath{\mathbf{R}}\xspace}	
% Real numbers
\newcommand{\Z}{\ensuremath{\mathbf{R}}\xspace}	% Integers
\newcommand{\Q}{\ensuremath{\mathbf{Q}}\xspace}	% Fractions
\renewcommand{\cal}{\mathcal}

\newcommand{\pr}{\ensuremath{\mathbf{Pr}}}
\newcommand{\PR}[2]{\ensuremath{\mathbf{Pr}_{#1} \left [ #2 \right ]}}
\newcommand{\ex}{\ensuremath{\mathbf{E}}}
\newcommand{\EX}[2]{\ensuremath{\mathbf{E}_{#1} \left [ #2 \right ]}}
\newcommand{\mat}[1]{\ensuremath{\boldsymbol{#1}}}
\newcommand{\vect}[1]{\ensuremath{\boldsymbol{#1}}}
\newcommand{\bvect}{\ensuremath{\boldsymbol{B}}}
\newcommand{\trans}[1]{\ensuremath{{#1}^{\scriptscriptstyle \mathsf{T}}}}
\newcommand{\norm}[1]{\ensuremath{\left \| #1 \right \|}}
\newcommand{\normsq}[1]{\ensuremath{\left \| #1 \right \|^2}}
\newcommand{\pseudo}[1]{\ensuremath{{#1}^{+}}}
\newcommand{\inv}[1]{\ensuremath{{#1}^{-1}}}
\newcommand{\etal}{\emph{et al.\!}}
\renewcommand{\th}{\ensuremath{^{\textrm{th}}}}
\renewcommand{\epsilon}{\varepsilon}
\newcommand{\NMI}{\ensuremath{\mathrm{NMI}}}
\def\CPP{{C\nolinebreak[4]\hspace{-.05em}\raisebox{.4ex}{\tiny\bf ++}}}

\newtheorem{lemma}{Lemma}
\newtheorem{theorem}{Theorem}
\newtheorem{corollary}{Corollary}
\newtheorem{observation}{Observation}
\newtheorem{proposition}{Proposition}

\theoremstyle{definition}
\newtheorem{definition}{Definition}
