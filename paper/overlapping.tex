Detecting overlapping communities in a network is a harder problem than detecting non overlapping communities, because we do not know beforehand to how many communities a node might belong. The parameter used for the LFR benchmark were the same as in the previous section: $1000$ and $5000$ nodes, \textit{small} and \textit{big} communities and exponents of $-2$ and $-1$ for the degree and community distribution. This time however we do not vary the mixing factor $\mu$ but the fraction of nodes that lie in more than one community. We made separate plots for a fixed mixing parameter of $0.1$ and $0.3$. This is again the same setup used by \cite{LF09}.

In Figures \ref{fig:no_iter_overlap_1000N} and \ref{fig:no_iter_overlap_5000N} we show our benchmark results for the overlapping case. In \cite{LF09} only one algorithm was able to detect overlapping communities, and for that reason we only compare our results wit \textit{Cfinder} \cite{PDFV05}. The main thing that can be noticed is, that we perform a lot better on LFR graphs for a mixing factor of $0.3$. In the most extreme case of $1000$ nodes, big communities and a mixing factor of $0.3$ the \textit{Cfinder} algorithm has $60$ percent NMI as its best result at zero percent overlapping nodes, and goes only down if we increase the fraction of overlapping nodes.

For $5000$ nodes the results are similar but not as extreme. In conclusion we can say that our algorithm seems to perform better on LFR graphs and offers a better runtime compared to that of the exponential worst-case runtime of \textit{Cfinder}.

In Figures \ref{fig:iter_overlap_1000N} \ref{fig:iter_overlap_5000N} we have provided plots for the iterative method and a improvement plot in Figure \ref{fig:compare_iter_overlap}. Iteration yields an improvement, but it is not as drastic as in the non overlapping case, around $10$ percent at best.