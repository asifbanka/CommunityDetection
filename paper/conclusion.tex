Our algorithm seems to work very well with around 6$\%$ seed nodes for the 
non-overlapping case and around 8$\%$ seed nodes for the overlapping case. 
For the non-overlapping case, we can work with a mixing factor of up 
to~$0.5$, whereas in the overlapping case a mixing factor of $0.3$ 
and with the overlapping fraction of around $20\%$. This of course 
suggests that our algorithm has a higher tolerance while detecting 
non-overlapping communities and needs either a ``well-structured''
network or a high seed node percentage for overlapping communities. 
None of this is really surprising. What is surprising is that such 
a simple algorithm manages to do so well at all.

An obvious question is whether it is possible to avoid the semi-supervised 
step completely, that is, avoid having the user to specify seed nodes 
for every community. One possibility is to initially use a clustering 
algorithm to obtain a first approximation of the communities in the network. 
The next step would be to pick seed nodes from among the communities thus found 
(without user intervention) and use our algorithm to obtain a refinement of the 
community structure. 


A possible extension of our algorithm would be to allow the user to interactively 
specify the seed nodes. The user initially supplies a set of seed nodes 
and allows the algorithm to find communities. The user then checks the 
quality of the output and, if dissatisfied with the results, can prompt the algorithm 
to correctly classify some more nodes that it had incorrectly classified in the current round. 
In effect, at the end of each round, the user supplies an additional set of seed nodes until the 
communities found out by the algorithm are accurate enough for the user. Such a tool 
might be useful for visualization.



We wish to point out that while the running time of our algorithm is 
$O(k \cdot m \cdot \log n)$, we do not know of any commercial solvers 
for SDD systems that run in $O(m \cdot \log n)$ time. Since we use the Cholesky 
factorization method from the \CPP\ Eigen Library, it is unlikely that our 
implementation would be able to handle very large networks. Recall that 
in Cholesky factorization, the matrix of coefficients $\mat{A}$ is decomposed 
as $\mat{L} \mat{D} \trans{\mat{L}}$, where $\mat{L}$ is lower triangular 
and $\mat{D}$ is diagonal, all of which takes $n^3/3$ operations making it 
prohibitively expensive for large networks (see, for instance~\cite{GvL13}). 
This is not a serious disadvantage since we expect that in the near future 
we would have commercial SDD solvers implementing the Speilman-Teng algorithm. 
It would then be interesting to see the size range of real networks our algorithm 
can handle. 


