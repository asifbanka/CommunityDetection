In what follows, we briefly describe the more important algorithms 
for community detection. We are particularly interested in the performance 
of these algorithms as reported in the study by Lancichinetti and 
Fortunato~\cite{LF09} on their LFR benchmark graphs. 

\paragraph{The Girvan-Newman algorithm.} 
One of the very first algorithms for detecting disjoint communities 
was due to Girvan and Newman~\cite{GN02, GN04}. Their 
algorithm starts with a network and iteratively removes edges based 
on a metric called \emph{edge betweenness}. The betweenness of an 
edge is defined as the number of shortest paths between vertex pairs 
that pass through that edge. After an edge is removed, betweenness 
scores are recalculated and an edge with the highest score is deleted. 
This procedure ends when the modularity of the resulting partition
reaches a maximum. Modularity is a measure that estimates the goodness 
of a partition by comparing the network with a so-called ``null model''
in which edges are added at random between the nodes of the network 
so as to preserve the degree sequence of the nodes. 

Formally, the \emph{modularity} of a partition is defined as:
\begin{equation}\label{eqn:modularity}
	Q = \frac{1}{2m} \sum_{i, j} \left ( A_{i j} - \frac{d_i d_j}{2m} \right ) \delta(i, j),
\end{equation}
where $A_{ij}$ represent the entries of the adjacency matrix of the network; $d_i$ is the 
degree of node $i$; $m$ is the number of edges in the network; and $\delta(i, j) = 1$ if nodes
$i$ and $j$ belong to the same set of the partition and $0$ otherwise. The term $d_i d_j / 2m$ 
represents the expected number of edges between nodes $i$ and $j$ if we consider a random model
in which each node $i$ has $d_i$ ``stubs'' and we are allowed to connect stubs at random to form edges. 
This is the null model against which the within-community edges of the partition is compared against.
The worst-case complexity of the Newman-Girvan algorithm is dominated by the time taken 
to compute the betweeness scores and is $O(m^2 n)$ for general graphs and $O(n^3)$ for sparse graphs.

\paragraph{The greedy algorithm for modularity optimization by Clauset \etal~\cite{CNM04}.}
This algorithm starts with each node being the sole member of a community of one, and 
repeatedly joins two communities whose amalgamation produces the largest increase in modularity. 
The algorithm makes use of efficient data structures and has a running time of $O(m \log^2 n)$, 
which for sparse graphs works out to $O(n \log^2 n)$. 

\paragraph{Fast Modularity Optimization by Blondel \etal} 
The algorithm of Blondel \etal~\cite{BGLL08} consists of two phases which are repeated iteratively. 
It starts out by placing each node in its own community and then locally optimizing the modularity 
in the neighborhood of each node. In the second phase, a new network is built whose nodes are the 
communities found out in the first phase. That is, communities are replaced by ``super-nodes''; the 
within-community edges are modeled by a (weighted) self-loop to the super-node; and the between-community 
edges are modeled by a single edge between the corresponding super-nodes, with the weight being 
the sum of the weights of the edges between these two communities. 
Once the second phase is complete, the entire procedure is repeated until the modularity does not 
increase any further. The algorithm is efficient due to the fact that one can quickly compute the 
change in modularity obtained by moving an isolated node into a community.

This is what Lancichinetti and Fortunato state about modularity-based techniques in their analysis of 
community detection algorithms~\cite{LF09}: 
\begin{quote}
	\emph{Modularity-based methods have a rather poor performance, which worsens for larger systems and smaller 
	communities due to the well-known resolution limit of the measure. The only exception is represented 
	by the algorithm of Blondel \emph{\etal}, whose performance is very good probably because the estimated modularity 
	is not a very good approximation of the real one $\ldots$} 
\end{quote}

\paragraph{The Cfinder algorithm of Palla \etal} One of the first algorithms that dealt with overlapping 
communities was proposed by Palla \etal~\cite{PDFV05}. A community, in this case, is defined to be a set of nodes 
that are the union of $k$-cliques such that any one clique can be reached from another via a series of 
adjacent $k$-cliques, where two $k$-cliques are defined to be adjacent if they 
share $k - 1$ nodes. 

The algorithm first finds out all maximal cliques in the graph, which takes exponential-time
in the worst case. It then creates a symmetric clique-clique overlap matrix $\mat{C}$ which is a square matrix whose 
rows and columns are indexed by the set of maximal cliques in the graph and whose $(i, j)\th$ 
is the number of vertices that are in both the $i\th$ and $j\th$ cliques. This matrix is then modified into 
a binary matrix by replacing all those entries with value less than $k - 1$ by a $0$ and the remaining entries by 
a $1$. The final step is to find the connected components of the graph represented by this binary symmetric matrix
which constitute the communities of the original graph. 

The authors report to have tested the algorithm on various networks including the protein-protein interaction 
network of \emph{Saccharomyces cerevisiae}\footnote{A species of yeast used in winemaking, baking, and brewing.} 
with $k = 4$; the co-authorship network of the Los Alamos condensed matter archive (with $k = 6$). 
Lancichinetti and Fortunato report that Cfinder did not perform particularly well on the LFR benchmark and 
that its performance is sensitive to the sizes of community (but not the network size). For networks will 
small communities it has a decent performance, but has a worse performance on those with larger communities. 

\paragraph{Using random walks to model information flow.} Rosvall and Bergman~\cite{RB08} approach the 
problem of finding communities from an information-theoretic angle. They transform the problem 
into one of finding an optimal description of an infinite random walk in the network.  Given a fixed 
partition $M$ of $n$ nodes into $k$ clusters, Rosvall and Bergman use a two-level code where each 
cluster is assigned a unique codeword but the codewords assigned to individual nodes within a 
cluster repeat between clusters. One can now define the average number of bits per step that is 
takes to describe an infinite random walk on the network partitioned according to $M$. The intuition 
is that a random walker is statistically likely to spend more time within clusters than between clusters
and therefore the ``best'' partition corresponds to the one which has the shortest possible description.
The best partition is found out using a combination of a greedy search heuristic followed by simulated 
annealing. Lancichinetti and Fortunato report that this algorithm (dubbed Infomap) was the best-performing 
among all other community detection algorithms on their benchmark.  

\paragraph{}
