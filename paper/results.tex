The results for non--overlapping networks can be seen in Figures~\ref{fig:no_iter_no_overlap},~\ref{fig:iter_no_overlap}, and~\ref{fig:compare_iter_no_overlap}. We have four plots per setup, using $5$, $10$, $15$ and $20$ percent of the nodes of the communities as seed nodes, which means we give the full community information provided by the LFR benchmark of these nodes as input. The percentages were chosen to show how much seed information is approximatively needed to get desirable results.

We start off with the batch of benchmarks are those for the non iterative method, and can be seen in Figure~\ref{fig:no_iter_no_overlap}. The first thing we notice is that five percent seed nodes are probably not sufficient to get desirable results, but if we get to at least ten percent we can, for a mixing factor of less or equal than $0.4$, already compete with \textit{Infomap} which was deemed one the best performing algorithm on LFR graphs by~\cite{LF09} and offer a discovery of communities with greater than $90$ percent accuracy. Above $0.4$ mixing factor we get worse get worse than \textit{Infomod}, which is able to keep its performance at $100$ percent NMI till a mixing factor of around $0.6$ but then drops of pretty steep. Our drop begins more early but is not as steep.\texttt{having a comparative plot for Infomod would be nice at this point.}

In Figure~\ref{fig:iter_no_overlap} we show benchmarks for the iterative approach of our algorithm. When compared with the non iterative results, we can see that after ten iterations we already get a huge improvement (See Figure~\ref{fig:compare_iter_no_overlap} for a direct comparison of the iterative method vs. non iterative). We can cut the amount of seed nodes in half and still get better results, if we use the iterative method. If we use the same number of seed nodes we can get NMI improvements as high as $60$ percent.\\ \\
In Figure~\ref{fig:no_iter_overlap_1000N} and~\ref{fig:no_iter_overlap_5000N} we show our benchmark results for the overlapping case. In~\cite{LF09} only one algorithm was able to detect overlapping communities, and for that reason we only compare our results wit \textit{Cfinder}~\cite{PDFV05}. The main thing that can be noticed is, that we perform a lot better on LFR graphs for a mixing factor of $0.3$. In the most extreme case of $1000$ nodes, big communities and a mixing factor of $0.3$ the \textit{Cfinder} algorithm has $60$ percent NMI as its best result at zero percent overlapping nodes, and goes only down if we increase the fraction of overlapping nodes.

For $5000$ nodes the results are similar but not as extreme. In conclusion we can say that our algorithm seems to perform better on LFR graphs and offers a better runtime compared to that of the exponential worst-case runtime of \textit{Cfinder}.

In Figure~\ref{fig:iter_overlap_1000N} and~\ref{fig:iter_overlap_5000N} we have provided plots for the iterative method and a improvement plot in~\ref{fig:compare_iter_overlap}. Iteration yields an improvement, but it is not as drastic as in the non overlapping case, around $10$ percent at best.