%width of the three types of plots
\newcommand{\plotwidth}{0.63\linewidth}
\newcommand{\cfinderwidth}{0.96\linewidth}
\newcommand{\otherplotswidth}{0.76\linewidth}

As in the last section, we first discuss our results for the non-overlapping case followed by 
the ones for the overlapping case.

\subsection{Non-overlapping communities}
Figures~\ref{fig:no_iter_no_overlap}, \ref{fig:iter_no_overlap}, and~\ref{fig:compare_iter_no_overlap}
show the plots that we obtained for non-overlapping communities. Figure~\ref{fig:no_iter_no_overlap}
shows tests for the non-iterative method of our algorithm with 5, 10, 15, and 20$\%$ seed nodes per 
community. The first observation here is that anything less than 10$\%$ seed nodes per community 
will probably not give good results. With a seed node percentage of 10$\%$ or more and 
a mixing factor of at most~$0.4$ we acheive an NMI above $0.9$ and can compete with \textit{Infomap}, 
which was deemed to be one the best performing algorithms on the LFR benchmark~\cite{LF09}. 
Above a mixing factor of $0.4$, our algorithm has a worse performance than \textit{Infomap} 
which, curiously enough, achieves an NMI of around 1 till a mixing factor of around 
$0.6$ after which its performance drops steeply. The drop in the pereformance of our algorithm 
begins earlier but is not as steep.\texttt{having a comparative plot for Infomap would be nice at this point.}

Figure~\ref{fig:iter_no_overlap} shows the results for the iterative approach of 
our algorithm in the non-overlapping case. When compared with the non-iterative approach, 
one can see that even after ten iterations there is a significant improvement in 
performance (See Figure~\ref{fig:compare_iter_no_overlap}). As can be seen, typically 
with 6$\%$ seed nodes per community we obtain acceptable performance (an NMI value of 
over $0.9$ with the mixing factor of up to $0.5$).  


\begin{figure}[h!]
    \centering
    \begin{subfigure}{0.5\textwidth}
    \centering
    \includegraphics[width=\plotwidth]{plots/nonoverlap_noniter_a.pdf}
    \end{subfigure}%
    \begin{subfigure}{0.5\textwidth}
    \centering
    \includegraphics[width=\plotwidth]{plots/nonoverlap_noniter_b.pdf}
    \end{subfigure}
    \begin{subfigure}{0.5\textwidth}
    \centering
    \includegraphics[width=\plotwidth]{plots/nonoverlap_noniter_c.pdf}
    \end{subfigure}%
    \begin{subfigure}{0.5\textwidth}
    \centering
    \includegraphics[width=\plotwidth]{plots/nonoverlap_noniter_d.pdf}
    \end{subfigure}
    \caption{Noniterative method for nonoverlapping communities.}\label{fig:no_iter_no_overlap}
\end{figure}
%
\begin{figure}[h!]
    \centering
    \begin{subfigure}{0.5\textwidth}
    \centering
    \includegraphics[width=\plotwidth]{plots/nonoverlap_iter_a.pdf}
    \end{subfigure}%
    \begin{subfigure}{0.5\textwidth}
    \centering
    \includegraphics[width=\plotwidth]{plots/nonoverlap_iter_b.pdf}
    \end{subfigure}
    \begin{subfigure}{0.5\textwidth}
    \centering
    \includegraphics[width=\plotwidth]{plots/nonoverlap_iter_c.pdf}
    \end{subfigure}%
    \begin{subfigure}{0.5\textwidth}
    \centering
    \includegraphics[width=\plotwidth]{plots/nonoverlap_iter_d.pdf}
    \end{subfigure}
    \caption{Iterative method for nonoverlapping communities.}\label{fig:iter_no_overlap}
\end{figure}
%
\begin{figure}[h]
    \centering
    \begin{subfigure}{0.5\textwidth}
    \centering
    \includegraphics[width=\plotwidth]{plots/nonoverlap_compare_a.pdf}
    \end{subfigure}%
    \begin{subfigure}{0.5\textwidth}
    \centering
    \includegraphics[width=\plotwidth]{plots/nonoverlap_compare_b.pdf}
    \end{subfigure}
    \caption{Comparison between the the iterative and non-iterative method for non-overlapping communities.}\label{fig:compare_iter_no_overlap}
\end{figure}
%
\begin{figure}[h!]
    \centering
    \begin{subfigure}{0.33\textwidth}
    \centering
    \includegraphics[width=\otherplotswidth]{lfrpaper/1_split_kropped.pdf}
    \end{subfigure}%
    \begin{subfigure}{0.33\textwidth}
    \centering
    \includegraphics[width=\otherplotswidth]{lfrpaper/2_split_kropped.pdf}
    \end{subfigure}%
    \begin{subfigure}{0.33\textwidth}
    \centering
    \includegraphics[width=\otherplotswidth]{lfrpaper/3_split_kropped.pdf}
    \end{subfigure}
    \begin{subfigure}{0.33\textwidth}
    \centering
    \includegraphics[width=\otherplotswidth]{lfrpaper/4_split_kropped.pdf}
    \end{subfigure}%
    \begin{subfigure}{0.33\textwidth}
    \centering
    \includegraphics[width=\otherplotswidth]{lfrpaper/5_split_kropped.pdf}
    \end{subfigure}%
    \begin{subfigure}{0.33\textwidth}
    \centering
    \includegraphics[width=\otherplotswidth]{lfrpaper/6_split_kropped.pdf}
    \end{subfigure}%
    \caption{
        Test of Infomap, CFinder, Clauset et al, Girvan Newman (GN) and Blonel et al 
        and the potts model by Ronhovde and Nussinov (RN) on the LFR benchmark for the nonoverlapping communities. 
        As usual, the NMI-value (y-axis) is plotted against the mixing factor (x-axis).
        Tests were performed on graphs with 1000 and 5000 nodes with big (B) and small (S) communities.
        These figures are taken from \texttt{name of the paper}.
    }
\end{figure}



\subsection{Overlapping communities}
Figures~\ref{fig:no_iter_overlap_1000N} and~\ref{fig:no_iter_overlap_5000N} 
show our results for the overlapping case. In the study of Lancichinetti and Fortunato~\cite{LF09}, 
only one algorithm (\emph{Cfinder}~\cite{PDFV05}) for overlapping communities was benchmarked. 
The main difference with the non-overlapping case is that typically our algorithm needs a larger 
seed node percentage per community. This is not surprising since in the overlapping case, we would 
need seed nodes from the various overlaps as well as from the non-overlapping portions of communities 
to make a good-enough calculation of the affinities. 

For graphs of both 1000 and 5000 nodes, our algorithm performs better 
than Cfinder upto an overlapping fraction of $0.4$. We stress that Cfinder 
has an exponential worst-case running time and would be infeasible on larger graphs. 
%
%The best case scenario for Cfinder The important fact is that our algorithm performs 
%a lot better on LFR benchmark graphs for 
%a mixing factor of $0.3$. In the most extreme case of $1000$ nodes, big communities 
%and a mixing factor of $0.3$ the \textit{CFinder} algorithm has $60$ percent NMI as 
%its best result at zero percent overlapping nodes, and goes only down if one increases 
%the fraction of overlapping nodes, whereas our algorithm starts at $80$ percent NMI 
%even for $5$ percent seed nodes. If we use at least $10$ percent we can go up to 
%an overlap of $0.3$ before we reach $60$ percent NMI.
%
%For $5000$ nodes the results are similar but not as extreme. In conclusion we can 
%say that our algorithm seems to perform better on LFR graphs and offers a better runtime 
%compared to that of the exponential worst-case runtime of \textit{CFinder}.

Figures~\ref{fig:iter_overlap_1000N} and~\ref{fig:iter_overlap_5000N} show the 
plots for the iterative method. A comparison of the non-iteratve and iterative method 
is shown in Figure~\ref{fig:compare_iter_overlap}. Iteration yields an improvement in performance,
as measured by the NMI, but it is not as dramatic as in the non-overlapping case 
with the NMI increase being at most $10\%$ at best. The percentage of seed nodes per community required in the iterative 
approach with a mixing factor of $0.3$ is around 8$\%$. 


\begin{figure}[h!]
    \centering
    \begin{subfigure}{0.5\textwidth}
    \centering
    \includegraphics[width=\plotwidth]{plots/overlap_noniter_1mu_a.pdf}
    \end{subfigure}%
    \begin{subfigure}{0.5\textwidth}
    \centering
    \includegraphics[width=\plotwidth]{plots/overlap_noniter_3mu_a.pdf}
    \end{subfigure}
    \begin{subfigure}{0.5\textwidth}
    \centering
    \includegraphics[width=\plotwidth]{plots/overlap_noniter_1mu_b.pdf}
    \end{subfigure}%
    \begin{subfigure}{0.5\textwidth}
    \centering
    \includegraphics[width=\plotwidth]{plots/overlap_noniter_3mu_b.pdf}
    \end{subfigure}
    \caption{Noniterative method for overlapping communities on 1000 nodes.}\label{fig:no_iter_overlap_1000N}
\end{figure}
%
\begin{figure}[h!]
    \centering
    \begin{subfigure}{0.5\textwidth}
    \centering
    \includegraphics[width=\plotwidth]{plots/overlap_noniter_1mu_c.pdf}
    \end{subfigure}%
    \begin{subfigure}{0.5\textwidth}
    \centering
    \includegraphics[width=\plotwidth]{plots/overlap_noniter_3mu_c.pdf}
    \end{subfigure}
    \begin{subfigure}{0.5\textwidth}
    \centering
    \includegraphics[width=\plotwidth]{plots/overlap_noniter_1mu_d.pdf}
    \end{subfigure}%
    \begin{subfigure}{0.5\textwidth}
    \centering
    \includegraphics[width=\plotwidth]{plots/overlap_noniter_3mu_d.pdf}
    \end{subfigure}
    \caption{Noniterative method for overlapping communities on 5000 nodes.}\label{fig:no_iter_overlap_5000N}
\end{figure}
%
\begin{figure}[h!]
    \centering
    \begin{subfigure}{0.5\textwidth}
    \centering
    \includegraphics[width=\plotwidth]{plots/overlap_iter_1mu_a.pdf}
    \end{subfigure}%
    \begin{subfigure}{0.5\textwidth}
    \centering
    \includegraphics[width=\plotwidth]{plots/overlap_iter_1mu_b.pdf}
    \end{subfigure}
    \begin{subfigure}{0.5\textwidth}
    \centering
    \includegraphics[width=\plotwidth]{plots/overlap_iter_3mu_a.pdf}
    \end{subfigure}%
    \begin{subfigure}{0.5\textwidth}
    \centering
    \includegraphics[width=\plotwidth]{plots/overlap_iter_3mu_b.pdf}
    \end{subfigure}
    \caption{Iterative method for overlapping communities on 1000 nodes.}\label{fig:iter_overlap_1000N}
\end{figure}
%
\begin{figure}[h!]
    \centering
    \begin{subfigure}{0.5\textwidth}
    \centering
    \includegraphics[width=\plotwidth]{plots/overlap_iter_1mu_c.pdf}
    \end{subfigure}%
    \begin{subfigure}{0.5\textwidth}
    \centering
    \includegraphics[width=\plotwidth]{plots/overlap_iter_1mu_d.pdf}
    \end{subfigure}
    \begin{subfigure}{0.5\textwidth}
    \centering
    \includegraphics[width=\plotwidth]{plots/overlap_iter_3mu_c.pdf}
    \end{subfigure}%
    \begin{subfigure}{0.5\textwidth}
    \centering
    \includegraphics[width=\plotwidth]{plots/overlap_iter_3mu_d.pdf}
    \end{subfigure}
    \caption{Iterative method for overlapping communities on 5000 nodes.}\label{fig:iter_overlap_5000N}
\end{figure}
%
\begin{figure}[h!]
    \centering
    \begin{subfigure}{0.5\textwidth}
    \centering
    \includegraphics[width=\plotwidth]{plots/overlap_compare_a.pdf}
    \end{subfigure}%
    \begin{subfigure}{0.5\textwidth}
    \centering
    \includegraphics[width=\plotwidth]{plots/overlap_compare_b.pdf}
    \end{subfigure}
    \caption{Comparison between the the iterative and non-iterative method for overlapping communities.}\label{fig:compare_iter_overlap}
\end{figure}
%
\begin{figure}[h!]
    \centering
    \begin{subfigure}{0.5\textwidth}
    \centering
    \includegraphics[width=\cfinderwidth]{lfrpaper/fig6.pdf}
    \end{subfigure}%
    \begin{subfigure}{0.5\textwidth}
    \centering
    \includegraphics[width=\cfinderwidth]{lfrpaper/fig7.pdf}
    \end{subfigure}%
    \caption{
        Test of CFinder on the LFR benchmark with overlapping communities.
        Tests were performed on graphs with 1000 and 5000 nodes.
        These figures are taken from \texttt{name of the paper}.
    }
\end{figure}


