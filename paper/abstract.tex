Complex networks can be typically broken down into groups or modules. Discovering 
this ``community structure'' is an important step in studying the large-scale 
structure of networks. Many algorithms have been proposed for community detection 
and benchmarks have been created to evaluate their performance. Typically algorithms 
for community detection either partition the graph (non-overlapping 
communities) or find node covers (overlapping communities). 

In this paper, we propose a particularly simple semi-supervised learning 
algorithm for finding out communities. In essence, given the community information of a small 
number of ``seed nodes'', the method uses random walks from the seed nodes 
to uncover the community information of the whole network. The algorithm runs 
in time $O(k \cdot m \cdot \log n)$, where $m$ is the number of edges; $n$ 
the number of links; and $k$ the number of communities in the network. 
In sparse networks with $m = O(n)$ and a constant number of communities, this 
running time is almost linear in the size of the network. Another important 
feature of our algorithm is that it can be used for either non-overlapping 
or overlapping communities. 

We test our algorithm using the LFR benchmark created by Lancichinetti, 
Fortunato, and Radicchi~\cite{LFR08} 
specifically for the purpose of evaluating such algorithms. Our algorithm 
can compete with the best of algorithms for both non-overlapping 
and overlapping communities as found in the comprehensive study of 
Lancichinetti and Fortunato~\cite{LF09}.
