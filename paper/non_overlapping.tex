In this section we want to present the results for detecting non overlapping communities on the LFR benchmark. This is the simpler case of our tests because we now that a node is in exactly one community. The parameters of the LFR benchmark are chosen exactly as in \cite{LF09}, so that we are able to compare our results with those given by Lancichinetti and Fortunato. The average degree of the networks is $20$, the maximum degree $50$, the minus exponent of the degree distribution is $2$ and that of the community size distribution is $1$.

For our main test we distinguish between \textit{small} and \textit{big} community sizes, which vary between $10 - 50$ and $20 - 100$ nodes respectively. We have two network sizes that we investigate: $1000$ and $5000$ nodes. We have one setup for every of the four combinations of network and community sizes. For every setup we create graphs with those parameters and a varying mixing factor $\mu$ (the percentage of edges that are leaving the communities) and compare the results from our algorithm with the ground truth given by the benchmark using NMI (Normalized Mutual Information). Every point in the plot is the average over $100$ iterations using the same parameters. 

The benchmarks mentioned above are to compare our results with most of those mentioned in \cite{LF09}. The \textit{infomap} algorithm \cite{RB08} and the \textit{fast modularity optimization} \cite{BGLL08} however were also tested on much larger graphs.
Since our algorithm also runs in essentially linear time in the number of edges we replicated these tests, using graphs with $50000$ and $100000$ nodes. The maximum degree was set to $200$ and the communities vary between $20$ and $1000$.

Because our algorithm needs seed nodes as input, we have four plots per setup, using $5$, $10$, $15$ and $20$ percent of the nodes of the communities as seed nodes, which means we give the full community information provided by the LFR benchmark of these nodes as input. The percentages were chosen to show how much seed information is approximatively needed to get desirable results.