Here we want to present our results for detecting non overlapping communites. This is the most simple case of community detection because we now exactly in how many communities a node is included, exactly one. The parameters of the LFR benchmark are chosen exactly as in (CITE: ``Community detection algorithms: a comparative analysis''), so that we are able to compare our results with those given by Lancichinetti and Fortunato. The average degree of the network is $20$, the maximum degree $50$, the minus exponent of the degree distribution is $2$ and that of the community size distribution is $1$.

We distinguish between \textit{small} and \textit{big} community sizes, which vary between $10 - 50$ and $20 - 100$ nodes respectively. We have two network sizes that we investigate: $1000$ and $5000$ nodes. We have one setup for every of the four combinations of network and community sizes. For every setup we create graphs with those parameters and a varying mixing factor $\mu$ (the percentage of edges that are leaving the communities) and compare the results from our algorithm with the ground truth given by the benchmark using NMI (Normalized Mutual Information). Every point in the plot is the avarage over $100$ iterations using the same parameters.

Because our algorithm also needs some seed nodes as input we have four plots per setup, using $5$, $10$, $15$ and $20$percent of the nodes of the network as seed nodes, which means we give the full community information provided by the LFR benchmark as input. These values were chosen to show how much seed information is aproximatley needed to get desirable results.