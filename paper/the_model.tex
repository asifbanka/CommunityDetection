We assume that the complex networks that we deal with are modeled as connected, undirected graphs.  
The algorithm receives as input a network and a set of nodes such that there is at least one node 
from each community that we are aiming to discover. These nodes are called \emph{seed nodes} and 
it is possible that a particular seed node belongs to multiple communities.

We say a node has \emph{high affinity} to a community if it belongs to it 
and \emph{low affinity} if it does not. Intermediate affinity values are 
possible and correspond to a partial belonging. The user specifies the affinities 
of the seed nodes for each of the communities. For all other nodes, called 
\emph{non-seed nodes}, we want deduce the affinity to each community using 
the information given by the seed nodes' affinities and the network structure. 
The main idea is that non-seed nodes should adopt the affinities of seed nodes 
within their close proximity. We define a proximity measure based on random walks: 
Each random walk starts at a non-seed node, traverses through the graph, and ends 
as soon as it reaches a seed node. The affinity of a non-seed node~$u$ for a given 
community is then the weighted sum of the affinities of the seed nodes for 
that community and reachable by a random walk starting at $u$, the weights 
being the probabilities that a random walk from $u$ ends up at a certain seed node.

Each step of a random walk can be represented as the iterated product of 
a transition matrix $\mat{P}$. The result of the (infinite) walk itself 
can be expressed as $\lim_{k \rightarrow \infty}{\mat{P}^k}$.
One of the contributions of this paper is to show how the calculation
of these limits can be reduced to solving a symmetric, diagonally dominant
system of linear equations (with different right-hand-sides per community),
which can be done in $O(m \log n)$ time, where~$m$ is the number of edges 
in the graph. The fact that such systems can be solved in almost linear time 
was discovered by Spielman and 
Teng~\cite{ST04,EEST05,ST08,KMP10,KMP11,Vis13}. If we assume that our networks are 
sparse in the sense that $m = O(n)$, the running time of our algorithm can  
be bounded by $O(n \log n)$.    



\subsection{Absorbing Markov Chains and Random Walks}
We now provide a formal description of our model. 
The input is an undirected, connected graph $G = (V,E)$ with nodes $v_1, \ldots, v_n$, 
$m$ edges and a nonempty set of $s$ seed nodes. We also know that there are~$k$ 
(possibly overlapping) communities which we want to discover.

The community information of a node~$v$ is represented 
by a $1 \times k$ vector called the \emph{affinity vector} of~$v$, denoted 
by 
\[
	\bvect (v) = \trans{\left ( \alpha (v, 1), \ldots, \alpha (v, k) \right )}.
\] 
The entry $\alpha (v, l)$ of the affinity vector represents the affinity of node~$v$ to community~$l$.
It may be interpreted as the probability that a node belongs to this community.
We point out that $\sum_{i = 1}^k \alpha (v, l)$ need not be~$1$.
An example of this situation is when~$v$ belongs to multiple 
communities with probability~$1$. 
The user-chosen affinity vectors of all seed nodes are part of the input.
The objective is to derive the affinity vectors of all non-seed nodes. 

\begin{figure}
\centering
\begin{subfigure}{\textwidth}
    \centering
    \begin{tikzpicture}
        [ auto
        , ->
        , >=stealth'
        , shorten >=1pt
        , node distance=2cm
        ]
        \node[mainNode] (1) {$x_1$};
        \node[mainNode] (2) [right of=1] {};
        \node[mainNode] (3) [right of=2] {};
        \node[mainNode] (4) [right of=3] {};
        \node[mainNode] (5) [right of=4] {$x_2$};

        \path[every node/.style={font=\sffamily\small}]
        (1) edge [bend right] node {} (2)
        (2) edge [bend right] node {} (3)
        (3) edge [bend right] node {} (4)
        (4) edge [bend right] node {} (5)

        (5) edge [bend right] node {} (4)
        (4) edge [bend right] node {} (3)
        (3) edge [bend right] node {} (2)
        (2) edge [bend right] node {} (1)
        ;
    \end{tikzpicture}
    \caption{Example graph with seed nodes $x_1$, $x_2$.}
\end{subfigure}
\\[0.5cm]
\begin{subfigure}{\textwidth}
    \centering
    \begin{tikzpicture}
        [ 
          auto
        , ->
        , >=stealth'
        , shorten >=1pt
        , node distance=2cm
        ]
        \node[mainNode] (1) {$x_1$};
        \node[mainNode] (2) [right of=1] {};
        \node[mainNode] (3) [right of=2] {};
        \node[mainNode] (4) [right of=3] {};
        \node[mainNode] (5) [right of=4] {$x_2$};

        \path[every node/.style={font=\sffamily\small}]
        (1) edge [loop left]  node {} (1)
        (5) edge [loop right] node {} (5)

        (2) edge [bend right] node {} (3)
        (3) edge [bend right] node {} (4)
        (4) edge node {} (5)

        (4) edge [bend right] node {} (3)
        (3) edge [bend right] node {} (2)
        (2) edge node {} (1)
        ;
    \end{tikzpicture}
    \caption{Example graph with seed nodes $x_1$, $x_2$ after transformation.}
\end{subfigure}
\caption[Removal of outgoing edges of seed nodes in a graph]
{Remove outgoing edges and add self-loop for all seed nodes in an example graph. A random walk reaching $x_1$ or $x_2$ will stay there forever.}
\label{fig:modgraph}.
\end{figure}

Since we require the random walks to end as soon as they reach a seed node, we transform 
the undirected graph~$G$ into a directed graph $G'$ as follows: replace each undirected 
edge $\{u, v\}$ by arcs $(u, v)$ and $(v, u)$; then for each seed node, remove its 
outarcs and add a self-loop. This procedure is illustrated in Figure~\ref{fig:modgraph}.


Random walks in this graph can be modelled by an $n \times n$ transition matrix $\mat{P}$, with
\begin{equation}\label{eqn:defining_prob}
	\mat{P}(i, j) = \left \{ 
							\begin{array}{ll}
                                \frac{1}{\deg_{G'} (v_i)} & \mbox{ if } (v_i, v_j) \in E(G') \\
								0			& \mbox{ otherwise,}
							\end{array}
					\right.
\end{equation}
where $\deg_{G'} (v)$ is the degree of node $v$ in the directed graph~$G'$.
The entry $\mat{P}(i, j)$ represents the transition-probability from node $v_i$ to $v_j$.
Additionally, $\mat{P}^r (i, j)$ may be interpreted 
as the probability that a random walk starting at node~$v_i$ will end up at node~$v_j$ 
after~$r$ steps. 

Assume that the nodes of~$G'$ are labeled $u_1, \ldots, u_{n - s}, x_{1}, \ldots, x_{s}$, 
where $u_1, \ldots, u_{n - s}$ are the non-seed nodes and $x_1, \ldots, x_s$ are the seed nodes. 
We can now write the transition matrix $\mat{P}$ in the following canonical form:
\begin{equation}\label{eqn:canonical_form_P}
	\mat{P} = 	\left [ \begin{array}{ll}
						\mat{Q}  & \mat{R} \\
						 \mat{0}_{s \times (n - s)} & \mat{I}_{s \times s}
						\end{array}
				\right ],
\end{equation}
where~$\mat{Q}$ is the $(n - s) \times (n - s)$ sub-matrix that represents the transition 
from non-seed nodes to non-seed nodes; $\mat{R}$ is the $(n - s) \times s$ sub-matrix 
that represents the transition from non-seed nodes to seed nodes. The $s \times s$ identity 
matrix $\mat{I}$ represents the fact that once a seed node is reached, one cannot transition away 
from it. Here $\mat{0}_{s \times (n - s)}$ represents an $s \times (n - s)$ matrix of zeros. 
Since each row of $\mat{P}$ sums up to~$1$ and all entries are positive, this matrix is stochastic.

It is well-known that such a stochastic matrix represents what is known as an 
\emph{absorbing Markov chain} (see, for example, Chapter~11 of Grinstead and Snell~\cite{GS98}).  
A Markov chain is called absorbing if it satisfies two conditions: 
It must have at least one absorbing state~$i$, where state~$i$ is defined 
to be absorbing if and only if $\mat{P}(i,i) = 1$ and $\mat{P}(i, j) = 0$ 
for all $j \neq i$. Secondly, it must be possible to transition from every state to 
some absorbing state in a finite number of steps.
It follows directly from the construction of the graph $G'$ and the fact that the 
original graph was connected, that random walks in $G'$ define an absorbing Markov chain. 
Here, the absorbing states correspond to the set of seed nodes.

For any non-negative $r$, one can easily show that:
\begin{equation}\label{exp:rth_product_of_P}
	\mat{P}^r = \left [ \begin{array}{ll}
						\mat{Q}^r  					& \sum_{i = 0}^{r - 1}\mat{Q}^i \cdot \mat{R} \\
						 \mat{0}_{s \times (n - s)} & \mat{I}_{s \times s}
						\end{array}
				\right ].
\end{equation}  
Since we are dealing with infinite random walks, we are interested in the following 
property of absorbing Markov chains.
\begin{proposition}\label{prop:limiting_Q}
	Let $\mat{P}$ be the $n \times n$ transition matrix that defines an absorbing Markov chain
	and suppose that $\mat{P}$ is in the canonical form specified by equation~(\ref{eqn:canonical_form_P}). 
	Then
    \begin{equation}
        \lim_{r \to \infty} \mat{P}^r = \left [ \begin{array}{ll}
            \mat{0}_{(n - s) \times (n - s)} & (\mat{I} - \mat{Q})^{-1} \cdot \mat{R} \\
            \mat{0}_{s \times (n - s)}       & \mat{I}_{s \times s}
        \end{array}
        \right ].
    \end{equation}  
\end{proposition}
Intuitively, every random walk starting at a non-seed node eventually 
reaches some seed node where it is ``absorbed.'' The probability 
that such an infinite random walk starting at non-seed node~$u_i$ ends 
up at the seed node~$x_j$ is entry $(i, j)$ of the 
submatrix $\mat{X} := (\mat{I} - \mat{Q})^{-1} \cdot \mat{R}$. 

Now we can finally define the affinity vectors of non-seed nodes.
The affinity of non-seed node $u_i$ to a community~$l$ is defined as:
\begin{equation}\label{eqn:belonging_vector}
    \alpha (u_i, l) = \sum_{j = 1}^{s} \mat{X} (i, j) \cdot \alpha (x_j, l).
\end{equation}
The computational complexity of calculating these affinity values  
depends on how efficiently we can calculate the entries of $\mat{X}$, 
i.e., solve $(\mat{I} - \mat{Q})^{-1}$. In the next subsection, we show 
how to reduce this problem to that of solving a system of linear equations 
of a special type which takes time $O(m \cdot \log n)$, where $m$
is the number of edges in $G$.

\subsection{Symmetric Diagonally Dominant Linear Systems}

An $n \times n$ matrix $\mat{A} = [a_{ij}]$ is \emph{diagonally dominant} if 
\[
	|a_{ii}| \geq \sum_{j \neq i} {|a_{ij}|} \mbox{ for all } i = 1, \ldots, n.
\] 
A matrix is \emph{symmetric diagonally dominant (SDD)} if, in addition to the above, 
it is symmetric. For more information about matrices and matrix computations, 
see the textbooks by Golub and Van Loan~\cite{GvL13} and Horn and Johnson~\cite{HJ13}. 

An example of a symmetric, diagonally dominant matrix is the graph Laplacian. 
Given an unweighted, undirected graph~$G$, the \emph{Laplacian} of $G$ 
is defined to be 
\[
\mat{L}_G = \mat{D}_G - \mat{A}_G,
\] 
where $\mat{A}_G$ is the adjacency matrix of the graph~$G$ and $\mat{D}_G$ 
is the diagonal matrix of vertex degrees. 

%This definition extends 
%to weighted graphs. A weighted graph $G = (V, E)$ has an associated weight function 
%$w \colon V \times V \to \R$ which satisfies the conditions: 
%\begin{inparaenum}[(1)]
%	\item $w(u, v) = w(v, u)$;  
%	\item $w(u, v) \geq 0$; and, 
%	\item if $\{u, v\} \notin E$ then $w(u, v) = 0$. 
%\end{inparaenum}
%In the context of weighted graphs, the degree~$\deg_G (u)$ of a vertex~$u$ is defined to be 
%\[\label{eqn:weighted_degree}
%	\deg_G (u) = \sum_{v \in V} w(u, v). 
%\] 
%The adjacency matrix $\mat{A}_{G}$ is defined as:
%\[\label{eqn:weighted_adj_matrix}
%	\mat{A}_{G}(u, v) = w(u, v),
%\]
%and the Laplacian is defined as before as: 
%\[\label{eqn:weighted_laplacian}
%	\mat{L}_G = \mat{D}_G - \mat{A}_G,
%\]
%where $\mat{D}_G$ is the diagonal matrix of vertex degrees in the weighted graph~$G$.

A symmetric, diagonally dominant (SDD) system of linear equations is a system of 
equations of the form:
\[
	\mat{A} \cdot \vect{x} = \vect{b},
\]
where $\mat{A}$ is an SDD matrix, $\vect{x} = \trans{(x_1, \ldots, x_n)}$ 
is a vector of unknowns, and $\vect{b} = \trans{(b_1, \ldots, b_n)}$ is a vector of constants. 
There is near-linear time algorithm for solving such a system of linear equations 
and this result is crucial to the analysis of the running time of our algorithm. 

The solution of $n \times n$ system of linear equations takes $O(n^3)$ time 
if one uses Gaussian elimination. Spielman and Teng made a seminal contribution in this direction and 
showed that SDD linear systems can be solved in nearly-linear 
time~\cite{ST04,EEST05,ST08}. Spielman and Teng's algorithm (the ST-solver)
iteratively produces a sequence of approximate solutions which converge to the 
actual solution of the system $\mat{A} \vect{x} = \vect{b}$. The performance 
of such an iterative system is measured in terms of the time taken to reduce 
an appropriately defined approximation error by a constant factor. The time 
complexity of the ST-solver was reported to be at least $O(m \log^{15} n)$~\cite{KMP11}.  
Koutis, Miller and Peng~\cite{KMP10,KMP11} developed a simpler and faster algorithm 
for finding $\epsilon$-approximate solutions to SDD systems in time 
$\tilde{O}(m \log n \log (1/\epsilon) )$, where the $\tilde{O}$ notation hides 
a factor that is at most $(\log \log n)^2$. A highly readable account 
of SDD systems is the monograph by Vishnoi~\cite{Vis13}. We summarize the 
main result that we use as a black-box.  
\begin{proposition} \label{prop:SDD_systems} {{\rm \cite{KMP11,Vis13}}}
	Given a system of linear equations $\mat{A} \vect{x} = \vect{b}$, where $\mat{A}$
	is an SDD matrix, there exists an algorithm to compute $\tilde{\vect{x}}$  
	such that:
		\[
			\norm{\tilde{\vect{x}} - \vect{x}}_{\mat{A}} \leq \epsilon \norm{\vect{x}}_{\mat{A}}, 
		\]
	where $\norm{\vect{y}}_{\mat{A}} := \sqrt{\trans{\vect{y}} \mat{A} \vect{y}}$. The algorithm runs in 
	time $\tilde{O}(m \cdot \log n \cdot \log (1 / \epsilon) )$ time, where $m$ is the number of non-zero 
	entries in $\mat{A}$. The $\tilde{O}$ notation hides a factor of at most $(\log \log n)^2$.
\end{proposition} 

%Given an $n \times n$ matrix~$\mat{A}$ and $\vect{b} \in \R^{n}$, consider the 
%system of linear equations $\mat{A} \vect{x} = \vect{b}$, with variables
%$\vect{x} = \trans{(x_1, \ldots, x_n)}$. Such a system has a solution if and only if
%$\vect{b}$ is in the image of matrix~$\mat{A}$. If $\mat{A}$ is invertible, then 
%the system has a solution for all $\vect{b} \in \R^n$. However, if $\vect{b}$ 
%is in the image of $\mat{A}$, then the inverse of $\mat{A}$ is well-defined 
%and is denoted by $\pseudo{\mat{A}}$. This is called the \emph{pseudo-inverse}
%of $\mat{D}$. 
%\begin{proposition} \label{prop:Laplacian_systems} {\textrm{\cite{ST08,Vis13}}}
%	There is an algorithm which takes a graph Laplacian $\mat{L}$, a vector 
%	$\vect{b}$, and an error parameter~$\epsilon > 0$, and returns $\vect{x}$
%	such that:
%		\[
%			\norm{\vect{x} - \pseudo{\mat{L}} \vect{b}}_{\mat{L}} \leq \epsilon \norm{\pseudo{\mat{L}} \vect{b}}_{\mat{L}}, 
%		\]
%	where $\norm{\vect{y}}_{\mat{L}} := \sqrt{\trans{\vect{y}} \mat{L} \vect{y}}$. The algorithm runs in 
%	time $O(m \cdot \log n \cdot \log 1 / \epsilon)$ time, where $m$ is the number of non-zero 
%	entries in $\mat{L}$.
%\end{proposition} 
%This result can be generalized to linear systems $\mat{A} \vect{x} = \vect{b}$, where 
%$\vect{b}$ is a symmetric (weakly) diagonally dominant matrix.

We can use Proposition~\ref{prop:SDD_systems} to upper-bound the time taken to solve
the linear systems, which are needed to calculate the affinity vectors defined in (\ref{eqn:belonging_vector}).

\begin{theorem}\label{theorem:computing_NR}
Given a graph~$G$, let $\mat{P}$ be the $n \times n$ transition matrix 
defined by equation~(\ref{eqn:defining_prob}) in canonical form 
(see equation~(\ref{eqn:canonical_form_P})). Then, one can compute 
the affinity vectors of all non-seed nodes in time $O(m \cdot \log n)$ per community, 
where~$m$ is the number of edges in the graph~$G$.
\end{theorem}  
\begin{proof}
Recall that we ordered the nodes of $G$ as $u_1, \ldots, u_{n - s}, x_1, \ldots, x_s$, 
where $u_1, \ldots, u_{n - s}$ denote the non-seed nodes and $x_1, \ldots, x_s$ denote 
seed nodes. Define $G_1 := G[u_1, \ldots, u_{n - s}]$, the subgraph induced by the non-seed nodes 
of~$G$. Let $\mat{A}_1$ denote the adjacency matrix of the graph $G_1$; let 
$\mat{D}_1$ denote the $(n - s) \times (n - s)$ diagonal matrix satisfying 
$\mat{D}_1(u_i, u_i) = \deg_{G}(u_i)$ for all $1 \leq i \leq n - s$.  That is, the 
entries of $\mat{D}_1$ are not the degrees of the vertices in the induced subgraph~$G_1$ 
but in the graph~$G$. We can then express 
$\mat{I} - \mat{Q}$ as 
\begin{equation} \label{eqn:I-Q}
	\mat{I}  - \mat{Q} = \inv{\mat{D}_1} (\mat{D}_1 - \mat{A}_1).
\end{equation}
Note that $\mat{D}_1 - \mat{A}_1$ is a symmetric and diagonally dominant matrix. 
Let us suppose that $\mat{X}$ is an $(n - s) \times s$ matrix such that 
\[
	\mat{X} = (\mat{I} - \mat{Q})^{-1} \cdot \mat{R}.
\]

Fix a community~$l$. Then the affinities of the non-seed nodes 
for community~$l$ may be written as:
\begin{align} \label{eqn:affinity}
	\left ( \begin{array}{c}
		\alpha (u_1, l) \\
		\vdots			\\
		\alpha (u_{n - s}, l)
	\end{array}	\right ) & = \sum_{j = 1}^{s} \alpha (x_j, l) \cdot \mat{X}_j \nonumber \\ 
						 & = \sum_{j = 1}^{s} \alpha (x_j, l) \inv{(\mat{I} - \mat{Q})} \cdot \mat{R}_j \nonumber \\ 
						 & = \inv{(\mat{I} - \mat{Q})} \cdot \sum_{j = 1}^{s} \alpha (x_j, l) \cdot \mat{R}_j,
\end{align}
where $\mat{X}_j$ and $\mat{R}_j$ denote the $j\th$ columns of $\mat{X}$ and $\mat{R}$, respectively. 
Using equation~(\ref{eqn:I-Q}), we may rewrite equation~(\ref{eqn:affinity}) as:
\begin{align}
	\inv{\mat{D}_1} (\mat{D}_1 - \mat{A}_1) \cdot \left ( \begin{array}{c}
		\alpha (u_1, l) \\
		\vdots			\\
		\alpha (u_{n - s}, l)
	\end{array}	\right ) & = \sum_{j = 1}^{s} \alpha (x_j, l) \cdot \mat{R}_j.
\end{align}
Finally, multiplying by $\mat{D}_1$ on both sides, we obtain
\begin{equation}\label{eqn:final_affinity}
	(\mat{D}_1 - \mat{A}_1) \cdot \vect{\alpha}_l = \mat{D}_1 \cdot \sum_{j = 1}^{s} \alpha (x_j, l) \cdot \mat{R}_j,
\end{equation}
where we used $\vect{\alpha}_l$ to denote the vector $\trans{\left ( \alpha (u_1, l), \ldots, \alpha (u_{n-s}, l) \right )}$.

Note that computing $\sum_{j = 1}^{s} \alpha (x_j, l) \cdot \mat{R}_j$ takes time $O(\tilde{m})$, where $\tilde{m}$ 
denotes the number of non-zero entries\footnote{This is almost the same as the 
number~$m$ of edges in $G$, but not quite, since while constructing $\mat{P}$ from the graph $G$, 
we add self-loops on seed nodes and delete edges between adjacent seed nodes, if any. However what is true is 
that $\tilde{m} \leq m + s \leq m + n$.} in \mat{P}. 
Computing the product of $\mat{D}_1$ and $\sum_{j = 1}^{s} \alpha (x_j, l) \cdot \mat{R}_j$ 
takes time $O(\tilde{m})$ so that the right hand side of equation~(\ref{eqn:final_affinity}) can 
be computed in time $O(\tilde{m})$. We now have a symmetric diagonally dominant system of linear equations 
which by Proposition~\ref{prop:SDD_systems} can be solved in time $O(\tilde{m} \cdot \log n)$. Therefore,
the time taken to compute the affinity to a fixed community is $O(\tilde{m} \cdot \log n) = O(m \log n)$,
which is what was claimed. Since we assume our networks to be sparse, $m = O(n)$, and 
the time taken is $O(n \cdot \log n)$ per community.  
%
%Then $(\mat{I} - \mat{Q}) \mat{X} = \mat{R}$, which may be rewritten as
%$(\mat{D}_1 - \mat{A}_1) \mat{X} = \mat{D}_1 \cdot \mat{R}$. The latter 
%equation may be written as a set of linear equation systems:
%\begin{equation}\label{eqn:laplacian_linear_eqn}
%	(\mat{D}_1 - \mat{A}_1) \cdot \vect{x}_i = \mat{D}_1 \cdot \vect{r}_i  \quad (\mbox{for } 1 \leq i \leq s),
%\end{equation} 
%where $\vect{x}_i$ and $\vect{r}_i$ are the $i\th$ columns of $\mat{X}$ and $\mat{R}$, respectively. 
%By Proposition~\ref{prop:SDD_systems}, the time taken to solve each of these systems 
%is $O(m \log n)$. Together, they can be solved in time $O(s \cdot m \cdot \log n)$, which
%is what was claimed.  
\end{proof}

