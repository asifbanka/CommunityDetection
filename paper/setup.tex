\begin{figure}
    \centering
    \begin{tikzpicture}[node distance=1cm, auto]  
        \tikzset{
            mynode/.style={
                rectangle,
                rounded corners,
                draw=black, 
                %top color=white, 
                %bottom color=yellow!50,
                very thick, 
                inner sep=1em, 
                minimum size=3em, 
                text centered
            },
            myarrow/.style={
                ->, 
                >=latex', 
                shorten >=1pt, 
                thick
            },
            mylabel/.style={
                text width=7em,
                text centered
            },
            % I found this nice trick at http://tex.stackexchange.com/questions/50780/arrows-at-right-angles-on-a-tikzpicture-matrix
            bendedarrow/.style={
                to path={(\tikztostart) -- ++(#1,0pt) \tikztonodes |- (\tikztotarget) },
                pos=0.5
            }
        }  
        
        % NODES

        \node[mynode]                  (lfr)            {LFR};  
        \node[mynode, below=of lfr]    (seeds)          {Seed Generation};  
        \node[mynode, below=of seeds]  (walk)           {Random Walk};  
        \node[mynode, below=of walk]   (classification) {Classification};  

        % shift nmi node to the right
        \coordinate (middle) at ($(seeds)!0.5!(walk)$);
        \node[mynode, right=of middle, xshift=2cm] (nmi) {NMI};  

        % EDGES

        % simple edges
        \draw[myarrow] (lfr)   -- (seeds);	
        \draw[myarrow] (seeds) -- (walk);	
        \draw[myarrow] (walk)  -- (classification);	

        % ingoing edges to nmi
        \draw[myarrow] (lfr)              -| (nmi);	
        \draw[myarrow] (classification)   -| (nmi);	

        % outgoing edge from nmi
        \node[right=1cm of nmi] (dummy1) {}; 
        \draw[myarrow] (nmi)   -- (dummy1);	

        % a dashed, bended arrow with iteration written on it
        \draw [myarrow, dashed, bendedarrow=-2.5cm] (classification) to (seeds);
        \node[left=1cm of walk] (dummy2) {}; 
        \node[mylabel, rotate=90] at (dummy2) (label1) {Iteration};  

    \end{tikzpicture} 
    \label{fig:pipeline}
    \caption{Pipeline} 
\end{figure}
