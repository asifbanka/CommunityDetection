\usepackage{acrocnym}
%
\begin{acronym}
	\acro{NMI}{normalized mutual information}
	\acro{LFR}Lancichinetti--Fortunato random graph model}
\end{acronym}
%
\section{Experimental Results}
Our experimental setup can be roughly divided into five parts (see \cref{fig:}. 
First, we prompt a benchmark graph generator developed by Lacichinetti and 
Fortunato which outputs random networks and its communities which are the 
ground truth. Then we pick a fixed amount of seed nodes and input it together 
with the other data into our algorithm. Thereafter we obtain the calculated 
affinities from each vertex to every respective community and start the 
community classification process. Last but not least, we compare the data by 
computing the \ac{NMI}\cite{DDDA05}.

The \ac{LFR} is an extension of the configuration graph model and enhances this 
concept by adding the possibility to have overlapping community vertices in the 
network. The authors designed their code with the intention to establish 
a standard benchmark suite for community--detection algorithms. For this reason 
we choose this benchmark for our experiments and set the parameters in the same 
fashion.

Once we obtain the ground truth we have to decide for a reasonable seed node 
picking strategy. Since our premise is that seed nodes are typically well known 
vertices who have a lot of neighbours i. e. are structurely important, we 
choose to pick a fixed percentage\footnote{but at least one each.} of nodes per 
community which have the highest degree.

The coded algorithm itself is a realization of the theoretical formulations we 
established in \cref{sec:}. Since initial test runs required an unrealistically 
high amount of seed nodes in order to get satisfying results in the overlaping 
case, we further enhanced our algorithm by implementing an iterative seed node 
picking strategy. After one run of the regular algorithm we try to increase the 
amount of seed nodes per community. We do so by multypling the cardinality of 
the respective seed set $\left S \right$ by a constant factor 
$d$\footnote{We typically set this value to $1.1$.} and subtract 
$\left S \right$ from this. Further, we check the highest affinity value 
$a < 1$ per vertex and include it to a set of potential new seed nodes for the 
respective community. After that we establish at most 
$\left S \right - d * \left S \right$ new seed nodes per community and rerun 
the algorithm. We repeat this process after we reach a fixed amount of 
iterations\footnote{Usually the amount of iterations was $10$ or $20$.}.

The algorithm outputs affinity values for every vertex community combination. 
these values range between zero and one and are the sole factor whether we 
assign a vertex to a community or not. At this point the strategy for community 
classification is different depending on whether we deal with overlapping 
comunities or not. In the latter case we know that a vertex is assigned to 
exactly one community. thus, we assign the community for which the respective 
vertex has the highest repsective affinity value. In the overlapping case we 
apply a slighlty more sophisticated strategy, since a vertex is maybe assigned 
to not just one but potentially to up to all communities. First, we assume that 
there are one or more ``high'' affinity values which would correspond to the 
case that a vertex is in favor to be assigned to a community. On the other hand,
 there are one or more ``low'' values in the other case. However, we can not 
make any more qualified statements about the actual values of the affinities, 
since these numbers differ greatly depending on the structure of the graph and 
on the amount fo communities a vertex belongs to. Therefore, we can only assume 
that there is some difference in high and low values. For this reason, we first 
sort the affinities in descending order. then we compare all subsequent pairs 
by computing the difference. Then we define the point where we found the 
largest difference as our border between low and high affinity values. Any 
asscoiated community left of this border is assigned to the repsective vertex, 
wheras any affinity value on the right hand side is considered as a low value 
and thus the repsective community is not assigned.

After we have assigned the communities to our vertices we now want to measure 
the quality of our predictions. As Lacichinetti and Fortunato 
suggest\cite{LF09} we determine the \ac{NMI} values of the predicted community 
assignment and the ground truth. The \ac{NMI} determines how similar the 
information in bits is