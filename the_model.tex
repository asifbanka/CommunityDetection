A \emph{social network} is a graph that is either directed or undirected. 
In this paper, we deal with social networks that are modeled as undirected 
graphs. In the context of community detection in social networks, we make 
the following assumptions. 

Firstly, we assume that we know what the 
network at hand represents. That is, we have a sense of context about the network.
The network could represent friendships as in the Facebook network; 
co-authorships as in the DBLP network; it could be a network of professionals 
such as LinkedIn, or a network of movie lovers such as IMDB. 
Secondly, we assume that we know what is it that constitutes a community. 
This is crucial because, for one, we do not define what a community is. 
We take it as being given as part of the 
input specification. While this may appear as an artifice, in many situations 
in real life, when we are required to find communities, we do in fact know what 
it is that we are searching for. For example, we might want to find the set of 
people with a given political leaning, or a set of people who love a particular 
genre of movie or a set of people with a specific skill-set and experience. 
In each of these cases, we have a sense of what it is that is required to be found. 

In broad brush-strokes, this is how our algorithm works.  It requires that we know
some members from each community that we are aiming to discover. We call these members 
\emph{seed nodes}. A seed node can belong to several communitites and our algorithm 
actually finds out overlapping communities in the network. The algorithm uses the 
partial community information and propagates it to the other nodes of the network 
by simulating a random walk from non-seed nodes to the set of seed nodes. 

Suppose that $x_1, \ldots, x_s$ are the seed nodes in the network and that a random walk starting at 
a non-seed node~$u$ reaches these nodes with probabilities $p_1, \ldots, p_s$. If out 
of these~$s$ seed nodes, $x_{i_1}, \ldots, x_{i_r}$ belong to a community~$C$, then 
the algorithm assigns a probability of $\sum_{j = 1}^r p_{i_j}$ for the event 
that~$u$ belongs to~$C$. For each community, we can now assign a probability for 
the event that~$u$ belongs to it. The communities that the algorithm 
assigns~$u$ are those for which the probabilities exceed a certain threshold. 

The random walk can be represented by a transition matrix and the walk itself can be simulated 
by multiplying this matrix with itself repeatedly. This, however, takes $O(n^3)$
time, where~$n$ is the number of nodes in the network. One of the contributions of 
this paper is to show how this can be reduced to solving a set of Laplacian system 
of linear equations (one set per seed node), which by the work of Speilman and 
Teng~\cite{Vis13} can be done in $O(m \log n)$ time, where~$m$ is the number of edges
in the graph. We assume that our networks are sparse in the sense that $m = O(n)$, 
so that the running time of our algorithm can  be bounded by $O(s n \log n)$.    
         
