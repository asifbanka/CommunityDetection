Complex networks can be typically broken down into clusters that are
more densely connected on the inside than between each other. Discovering this ``community
structure'' is an important step in studying the large-scale structure of
networks. Many algorithms have been proposed for community detection and
benchmarks have been created to evaluate their performance. Typically algorithms
for community detection either find a partition of the node set of the graph
(non-overlapping communities) or find a collection of sets that cover the entire
node set (overlapping communities). 

In this paper, we propose a particularly simple semi-supervised learning
algorithm for finding out communities. In essence, given the community
information of a small number of \emph{seed nodes}, the method uses random walks
from the seed nodes to uncover the community information of the remaining
network.  Not only does our approach provide a uniform method to deal with both the 
non-overlapping and overlapping case, but it runs in almost linear time on sparse 
networks. More specifically, the algorithm runs in time $O(k \cdot m \cdot \log n)$, 
where $m$ is the number of edges; $n$ the number of nodes; and $k$ the number of communities
in the network.  

We test our algorithm using the LFR benchmark created by Lancichinetti,
Fortunato, and Radicchi~\cite{LFR08} specifically for the purpose of evaluating
such algorithms. Our algorithm can compete with the best of algorithms for both
non-overlapping and overlapping communities as found in the comprehensive study
of Lancichinetti and Fortunato~\cite{LF09}. Some of the algorithms in that study
include several variants of the well-known modularity maximization approach,
which takes $O(n^2)$ time on sparse graphs~\cite{RCC04}; clique detection
algorithms such as Cfinder which takes exponential time in the worst
case~\cite{PDFV05}; techniques based on random walks such as the Infomap method
which reportedly perform well in practice but whose  worst-case running times
have not been reported~\cite{RB08}. 
