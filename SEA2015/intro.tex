Many real-world graphs that model complex systems exhibit an organization 
into subgraphs, or \textit{communities} that are more densely connected on the inside than between each other. 
Social networks such as Facebook and LinkedIn divide into groups of friends, 
coworkers, or business partners; scientific collaboration networks divide 
themselves into research communities; the World Wide Web divides into groups 
of related webpages. The nature and number of communities provide 
a useful insight into the structure and organization of networks. 

Discovering the community structure of networks is an 
important problem in network science and is the subject 
of intensive research~\cite{GN02,GN04,CNM04,RCC04,DM04,PDFV05,NL07,BGLL08,RB08,RN09}. 
Existing community detection algorithms are 
distinguished by whether they find partitions of the node set 
(non-overlapping communities) or node covers (overlapping communities). 
Typically finding overlapping communities is a much harder problem and most of the 
earlier community detection algorithms focused on finding disjoint 
communities. A comparative analysis of several community detection algorithms 
(both non-overlapping and overlapping) was presented by Lancichinetti and Fortunato~\cite{LF09}. In this paper we closely follow their test framework, 
also called the LFR-benchmark.

The notion of a community is a loose one and currently there is no 
well-accepted definition of this concept. A typical approach is to define an 
objective function on the partitions of the node set of the network 
in terms of two sets of edge densities: the density of the 
edges within a partite set (intra-community edges) and the density of edges across partitions 
(inter-community edges). The ``correct'' partition is the one that maximizes this 
function. Various community detection algorithms formalize this
informal idea differently. One of the very first algorithms by
Girvan and Newman~\cite{GN02} introduced a measure known as \textit{modularity}
which, given a partition of the nodes of the network, compares the fraction of 
inter-community edges with the edges that would be present had they been 
rewired randomly preserving the node degrees. Other authors such as Palla 
\etal~\cite{PDFV05} declare communities as node sets that formed 
by overlapping maximal cliques. Rosvall and Bergstrom~\cite{RB08} 
define the goodness of a partition in terms of the number of bits required to 
describe per step of an infinite random walk in the network, the intuition being 
that in a ``correct'' partition, a random walker is likely to spend more time 
within communities rather than between communities, thereby decreasing the 
description of the walk.  

\paragraph{Our contribution.}
Given that it is unlikely that users of community detection algorithms 
would unanimously settle on one definition of what constitutes a community, 
we feel that existing approaches ignore the \emph{user perspective}.
To this end, we chose to design an algorithm that takes the network structure 
as well as user preferences into account. 
The user is expected to classify a small set of nodes of the network 
into communities (which may be 6--8\% of the nodes of each community).
Obviously this is possible only when the user has some 
information about the network, such as its semantics, which nodes 
are important and into which communities they are classified. 

Such situations are actually quite common. The user might have data only on the
leading scientific authors in a co-authorship network and would like to find out
the research areas of the remaining members of the network.  He may either be
interested in a broad partition of the network into its main fields or a
fine grained decomposition into various subfields.  By labeling the known
authors accordingly, the user can specify which kind of partition he is
interested in.  Another example would be the detection of trends in a social
network.  Consider the case where one knows the political affiliations of some
people and aims to discover political spectrum of the whole network, for
example, to predict the outcome of an election. 

There are two main advantages of our approach: Firstly, it can be used to find
not only non-overlapping communities, but overlapping communities as well. Of
course, there is a higher price that has to be paid in the overlapping
communities case in that the number of nodes that need to be classified by the
user is typically larger (5\% to 10\% of the nodes per community). The
algorithm, however, does not need any major changes and we view this is as an
aesthetically pleasing feature of our approach.  Secondly, our algorithm runs in
time $O(k \cdot m \cdot \log n)$, where $k$ is the number of communities to be
discovered (supplied by the user), $n$ and $m$ are the number of nodes and edges
in the network. In the case of sparse graphs and a constant number of
communities, the running time is $O(n \cdot \log n)$.  Compare this with
algorithms that optimize modularity which take $O(n^2)$ even on sparse
networks. The overlapping clique finding algorithm of Palla \etal~\cite{PDFV05}
takes exponential time in the worst case. We evaluate our algorithms on the LFR
benchmark and in order to ensure a fair comparison with other algorithms
reviewed in the study of Lancichinetti and Fortunato~\cite{LF09}, we choose all parameters of the benchmark as in the original paper.

This paper is organized as follows. In Section~\ref{sec:algorithm}, we 
describe our algorithm and analyze its running time. In Sections~\ref{sec:experiment_setup} 
and~\ref{sec:experiment_results}, we present our experimental results. Finally we conclude 
in Section~\ref{sec:conclusions} with possibilities of how our approach might be extended. 



