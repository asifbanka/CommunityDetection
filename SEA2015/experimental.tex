Our experimental setup consists of five parts (see Figure~\ref{fig:pipeline})
but the respective parts differ slightly depending on whether we test
non-overlapping or overlapping communities. We use the LFR benchmark graph
generator developed by Lancichinetti, Fortunato, and Radicchi~\cite{LFR08,LF09},
which outputs graphs where the community information of each node is known. From
each community in the graph thus generated, we pick a fixed number of seed nodes
per community and give these as input to our algorithm.  Once the algorithm
outputs the affinities of all non-seed nodes, we classify them into communities
and finally compare the output with the ground truth using normalized mutual
information (NMI) as a metric~\cite{DDDA05}. We implemented our algorithm in
\CPP\ and Python and the code is available online.\footnote{At
\texttt{https://github.com/somnath1077/CommunityDetection}}


\paragraph{LFR.}
We briefly describe the major parameters that the user has to supply for
generating benchmark graphs in the LFR suite. The node degrees and the community
sizes are distributed according to a power law, with different exponents.  An
important parameter is the \emph{mixing parameter~$\mu$} which is the fraction
of neighbors of a node that do not belong to any community that the node belongs
to, averaged over all nodes.  The other parameters include maximum node degree,
average node degree, minimum and maximum community sizes. For generating
networks with overlapping communities, one can specify what fraction of nodes
are present in multiple communities.


In what follows, we describe tests for non-overlapping and overlapping
communities separately, since there are several small differences in our setup
for these two cases. 

\subsection{Non-overlapping communities}
The networks we test have either 1000 nodes or 5000 nodes. The average node
degree was set at 20 and the maximum node degree at 50. The parameter
controlling the distribution of node degrees was set at~2 and that for the
community size distribution was set at~1. Moreover, we distinguished between big
and small communities: small communities have 10--50 nodes and large communities
have 20--100 nodes.  For each of the four combinations of network and community
size, we generated graphs with the above parameters and with varying mixing
parameters. For each of these graphs, we tested the community information output
by our algorithm and compared it against the ground truth using the normalized
mutual information as a metric. The plots in the next section show how the
performance varies as the mixing parameter was changed. Each data point in these
plots is the average over 100 iterations using the same parameters. 

Due to lack of space, the exact details of how we generate seed nodes, classify
nodes into communities and iteratively improve the classification is relegated
to the appendix. 

\subsection{Overlapping Communities.}
The LFR benchmark suite can generate networks with an overlapping community
structure.  In addition to the parameters mentioned for the non-overlapping
case, there is an additional parameter that controls what fraction of nodes of
the network are in multiple communities.  As in the non-overlapping case, we
generated graphs with 1000 and 5000 nodes with the average node degree set at 20
and maximum node degree set at 50. We generated graphs with two types of
community sizes: small communities with 10--50 nodes and large communities with
20--100 nodes.  Moreover, as Lancichinetti and Fortunato~\cite{LF09}, we chose
two values for the mixing factor: $0.1$ and $0.3$ and we plot the quality of the
community structure output by the algorithm (measured by the NMI) against the
fraction of overlapping nodes in the network.

As in the non-overlapping case, the details of seed node generation, community
classification and the iterative technique to improve the classification are
dealt with in the appendix.


