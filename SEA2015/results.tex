%width of the three types of plots
\newcommand{\plotwidth}{0.8\linewidth}
\newcommand{\cfinderwidth}{0.96\linewidth}
\newcommand{\otherplotswidth}{0.76\linewidth}

As in the last section, we first discuss our results for the non-overlapping case followed by 
the ones for the overlapping case.

\subsection{Non-overlapping communities}
Figure~\ref{fig:main_iter_no_overlap} shows the plots that we obtained for non-overlapping communities using the iterative technique with 2, 4, 6, and 8$\%$ seed nodes per 
community on graphs with 1000 and 5000 nodes. The obtained NMI values are always after ten iterations of our algorithm.

As can be seen, with a seed node percentage of 6$\%$ or more and 
a mixing factor below~$0.5$ we achieve an NMI above $0.9$ and can compete with \textit{Infomap}, 
which was deemed to be one of the best performing algorithms on the LFR benchmark~\cite{LF09}. 
Above a mixing factor of $0.5$, our algorithm has a worse performance than \textit{Infomap} 
which, curiously enough, achieves a NMI of around 1 till a mixing factor of around 
$0.6$ after which its performance drops steeply. The drop in the performance of our algorithm 
begins earlier but is not as steep. See Figure~\ref{fig:main_Infomap_etal} for the performance 
of Infomap and other algorithms.

For more more detailed results including the non iterative method we refer to the Appendix
 
 


\begin{figure}[h!]
    \centering
    \begin{subfigure}{0.55\textwidth}
    \centering
    \includegraphics[width=\plotwidth]{plots/nonoverlap_iter_a.pdf}
    \end{subfigure}%
    \begin{subfigure}{0.55\textwidth}
    \centering
    \includegraphics[width=\plotwidth]{plots/nonoverlap_iter_b.pdf}
    \end{subfigure}
    \begin{subfigure}{0.55\textwidth}
    \centering
    \includegraphics[width=\plotwidth]{plots/nonoverlap_iter_c.pdf}
    \end{subfigure}%
    \begin{subfigure}{0.55\textwidth}
    \centering
    \includegraphics[width=\plotwidth]{plots/nonoverlap_iter_d.pdf}
    \end{subfigure}
    \caption{Iterative method for non-overlapping communities.}\label{fig:main_iter_no_overlap}
\end{figure}

\begin{figure}[h!]
    \centering
    \begin{subfigure}{0.35\textwidth}
    \centering
    \includegraphics[width=\otherplotswidth]{lfrpaper/1_split_kropped.pdf}
    \end{subfigure}%
    \begin{subfigure}{0.35\textwidth}
    \centering
    \includegraphics[width=\otherplotswidth]{lfrpaper/2_split_kropped.pdf}
    \end{subfigure}%
    \begin{subfigure}{0.35\textwidth}
    \centering
    \includegraphics[width=\otherplotswidth]{lfrpaper/3_split_kropped.pdf}
    \end{subfigure}
    \begin{subfigure}{0.35\textwidth}
    \centering
    \includegraphics[width=\otherplotswidth]{lfrpaper/4_split_kropped.pdf}
    \end{subfigure}%
    \begin{subfigure}{0.35\textwidth}
    \centering
    \includegraphics[width=\otherplotswidth]{lfrpaper/5_split_kropped.pdf}
    \end{subfigure}%
    \begin{subfigure}{0.35\textwidth}
    \centering
    \includegraphics[width=\otherplotswidth]{lfrpaper/6_split_kropped.pdf}
    \end{subfigure}%
    \caption{
        Plots for Infomap, CFinder, the algorithm of Clauset \etal, Girvan-Newman (GN), Blondel \etal, 
        and the Pott's model approach by Ronhovde and Nussinov (RN) on the LFR benchmark for non-overlapping 
		communities. As usual, the NMI-value ($y$-axis) is plotted against the mixing factor ($x$-axis).
        Tests were performed on graphs with 1000 and 5000 nodes with big (B) and small (S) communities.
        Reproduced from Lancichinetti and Fortunato~\cite{LF09}.
    }\label{fig:main_Infomap_etal}
\end{figure}

\begin{figure}[h!]
    \centering
    \begin{subfigure}{0.55\textwidth}
    \centering
    \includegraphics[width=\plotwidth]{plots/overlap_iter_1mu_c.pdf}
    \end{subfigure}%
    \begin{subfigure}{0.55\textwidth}
    \centering
    \includegraphics[width=\plotwidth]{plots/overlap_iter_1mu_d.pdf}
    \end{subfigure}
    \begin{subfigure}{0.55\textwidth}
    \centering
    \includegraphics[width=\plotwidth]{plots/overlap_iter_3mu_c.pdf}
    \end{subfigure}%
    \begin{subfigure}{0.55\textwidth}
    \centering
    \includegraphics[width=\plotwidth]{plots/overlap_iter_3mu_d.pdf}
    \end{subfigure}
    \caption{Iterative method for overlapping communities on 5000 nodes.}\label{fig:main_iter_overlap_5000N}
\end{figure}

\begin{figure}[h!]
    \centering
    \begin{subfigure}{0.5\textwidth}
    \centering
    \includegraphics[width=\cfinderwidth]{lfrpaper/fig6.pdf}
    \end{subfigure}%
    \begin{subfigure}{0.5\textwidth}
    \centering
    \includegraphics[width=\cfinderwidth]{lfrpaper/fig7.pdf}
    \end{subfigure}%
    \caption{
        Plots for CFinder on the LFR benchmark on graphs with 1000 and 5000 nodes 
		with overlapping communities. The parameters of the above row correspond to our \textit{small} and those of the bottom row to our \textit{big} communities Reproduced from Lancichinetti and Fortunato~\cite{LF09}.
    }\label{fig:main_CFinder_overlapping}
\end{figure}



\subsection{Overlapping communities}
Figure~\ref{fig:main_iter_overlap_5000N} shows our results for the overlapping case using the iterative method on graphs with 5000 nodes. Again, the results are obtained after ten iterations.

The main difference with the non-overlapping case is that typically our algorithm needs a larger 
seed node percentage per community. This is not surprising since in the overlapping case, we would need seed nodes from the various overlaps as well as from the non-overlapping portions of communities to make a good-enough calculation of the affinities. The percentage of seed nodes per community required in the iterative approach with a mixing factor of $0.3$ is around 8$\%$.  

In the study of Lancichinetti and Fortunato~\cite{LF09}, 
only one algorithm (\emph{Cfinder}~\cite{PDFV05}) for overlapping communities was benchmarked 
(see Figure~\ref{fig:main_CFinder_overlapping}). 

For graphs of both 1000 and 5000 nodes, our algorithm performs better 
than Cfinder up to an overlapping fraction of $0.4$. We stress that Cfinder 
has an exponential worst-case running time and would be infeasible on larger graphs. More detailed results can again be found in the Appendix. 





