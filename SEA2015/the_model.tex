We assume that the complex networks that we deal with are modeled as connected,
undirected graphs.  The algorithm receives as input a network $G = (V,E)$ and a
set $S \subseteq V$ of nodes such that there is at least one node from each
community that we are aiming to discover. These nodes are called \emph{seed
nodes} and it is possible that a particular seed node belongs to multiple
communities.

The affinity of a node in the network to a community is 1 if it belongs to it;
if it does not belong to it, it has an affinity of 0. We allow intermediate
affinity values and view these as specifying a \emph{partial belonging}. The
user specifies the affinities of the seed nodes for each of the communities. For
all other nodes, called \emph{non-seed nodes}, we want deduce the affinity to
each community using the information given by the seed nodes' affinities and the
network structure.  The main idea is that non-seed nodes should adopt the
affinities of seed nodes within their close proximity. We define a proximity
measure based on random walks: Each random walk starts at a non-seed node,
traverses through the graph, and ends as soon as it reaches a seed node. The
affinity of a non-seed node~$u$ for a given community is then the weighted sum
of the affinities of the seed nodes for that community and reachable by a random
walk starting at $u$, the weights being the probabilities that a random walk
from $u$ ends up at a certain seed node.

Each step of a random walk can be represented as the iterated product of a
transition matrix $\mat{P}$. The result of the (infinite) walk itself can be
expressed as $\lim_{k \rightarrow \infty}{\mat{P}^k}$.  One of the contributions
of this paper is to show how the calculation of these limits can be reduced to
solving a symmetric, diagonally dominant system of linear equations (with
different right-hand-sides per community), which can be done in $O(m \log n)$
time, where~$m$ is the number of edges in the graph. The fact that such systems
can be solved in almost linear time was discovered by Spielman and
Teng~\cite{ST04,EEST05,ST08,KMP10,KMP11,Vis13}. If we assume that our networks
are sparse in the sense that $m = O(n)$, the running time of our algorithm can
be bounded by $O(n \log n)$.    



\subsection{Absorbing Markov Chains and Random Walks}
We now provide a formal description of our model.  The input is an undirected,
connected graph $G = (V,E)$ with nodes $v_1, \ldots, v_n$, $m$ edges and a
nonempty set $S$ of $s$ seed nodes. We also know that there are~$k$ (possibly
overlapping) communities that we aim to discover.

The community information of a node~$v$ is represented 
by a $1 \times k$ vector called the \emph{affinity vector} of~$v$, denoted 
by 
\[
	\bvect (v) = \trans{\left ( \alpha (v, 1), \ldots, \alpha (v, k) \right )}.
\] 
The entry $\alpha (v, l)$ of the affinity vector represents the affinity of
node~$v$ to community~$l$.  It may be interpreted as the probability that a node
belongs to this community.  We point out that $\sum_{i = 1}^k \alpha (v, l)$
need not be~$1$.  An example of this situation is when~$v$ belongs to multiple
communities with probability~$1$.  The user-chosen affinity vectors of all seed
nodes are part of the input.  The objective is to derive the affinity vectors of
all non-seed nodes. 

\begin{figure}
\centering
\begin{subfigure}{\textwidth}
    \centering
    \begin{tikzpicture}
        [ auto
        , ->
        , >=stealth'
        , shorten >=1pt
        , node distance=2cm
        ]
        \node[mainNode] (1) {$x_1$};
        \node[mainNode] (2) [right of=1] {};
        \node[mainNode] (3) [right of=2] {};
        \node[mainNode] (4) [right of=3] {};
        \node[mainNode] (5) [right of=4] {$x_2$};

        \path[every node/.style={font=\sffamily\small}]
        (1) edge [bend right] node {} (2)
        (2) edge [bend right] node {} (3)
        (3) edge [bend right] node {} (4)
        (4) edge [bend right] node {} (5)

        (5) edge [bend right] node {} (4)
        (4) edge [bend right] node {} (3)
        (3) edge [bend right] node {} (2)
        (2) edge [bend right] node {} (1)
        ;
    \end{tikzpicture}
    \caption{Example graph with seed nodes $x_1$, $x_2$.}
\end{subfigure}
\\[0.5cm]
\begin{subfigure}{\textwidth}
    \centering
    \begin{tikzpicture}
        [ 
          auto
        , ->
        , >=stealth'
        , shorten >=1pt
        , node distance=2cm
        ]
        \node[mainNode] (1) {$x_1$};
        \node[mainNode] (2) [right of=1] {};
        \node[mainNode] (3) [right of=2] {};
        \node[mainNode] (4) [right of=3] {};
        \node[mainNode] (5) [right of=4] {$x_2$};

        \path[every node/.style={font=\sffamily\small}]
        (1) edge [loop left]  node {} (1)
        (5) edge [loop right] node {} (5)

        (2) edge [bend right] node {} (3)
        (3) edge [bend right] node {} (4)
        (4) edge node {} (5)

        (4) edge [bend right] node {} (3)
        (3) edge [bend right] node {} (2)
        (2) edge node {} (1)
        ;
    \end{tikzpicture}
    \caption{Example graph with seed nodes $x_1$, $x_2$ after transformation.}
\end{subfigure}
\caption[Removal of outgoing edges of seed nodes in a graph]
{Remove outgoing edges and add self-loop for all seed nodes in an example graph. A random walk reaching $x_1$ or $x_2$ will stay there forever.}
\label{fig:modgraph}.
\end{figure}

Since we require the random walks to end as soon as they reach a seed node, we transform 
the undirected graph~$G$ into a directed graph $G'$ as follows: first replace each undirected 
edge $\{u, v\}$ by arcs $(u, v)$ and $(v, u)$; then for each seed node, remove its 
outarcs and add a self-loop. This procedure is illustrated in Figure~\ref{fig:modgraph}.
Random walks in this graph can be modelled by an $n \times n$ transition matrix $\mat{P}$, with
\begin{equation}\label{eqn:defining_prob}
	\mat{P}(i, j) = \left \{ 
							\begin{array}{ll}
                                \frac{1}{\deg_{G'} (v_i)} & \mbox{ if } (v_i, v_j) \in E(G') \\
								0			& \mbox{ otherwise,}
							\end{array}
					\right.
\end{equation}
where $\deg_{G'} (v)$ is the degree of node $v$ in the directed graph~$G'$.
The entry $\mat{P}(i, j)$ represents the transition-probability from node $v_i$ to $v_j$.
Additionally, $\mat{P}^r (i, j)$ may be interpreted as the probability that a random 
walk starting at node~$v_i$ will end up at node~$v_j$ after~$r$ steps. 

Assume that the nodes of~$G'$ are labeled $u_1, \ldots, u_{n - s}, x_{1}, \ldots, x_{s}$, 
where $u_1, \ldots, u_{n - s}$ are the non-seed nodes and $x_1, \ldots, x_s$ are the seed nodes. 
We can now write the transition matrix $\mat{P}$ in the following canonical form:
\begin{equation}\label{eqn:canonical_form_P}
	\mat{P} = 	\left [ \begin{array}{ll}
						\mat{Q}  & \mat{R} \\
						 \mat{0}_{s \times (n - s)} & \mat{I}_{s \times s}
						\end{array}
				\right ],
\end{equation}
where~$\mat{Q}$ is the $(n - s) \times (n - s)$ sub-matrix that represents the transition 
from non-seed nodes to non-seed nodes; $\mat{R}$ is the $(n - s) \times s$ sub-matrix 
that represents the transition from non-seed nodes to seed nodes. The $s \times s$ identity 
matrix $\mat{I}$ represents the fact that once a seed node is reached, one cannot transition away 
from it. Here $\mat{0}_{s \times (n - s)}$ represents an $s \times (n - s)$ matrix of zeros. 
Since each row of $\mat{P}$ sums up to~$1$ and all entries are positive, this matrix is stochastic.

It is well-known that such a stochastic matrix represents what is known as an 
\emph{absorbing Markov chain} (see, for example, Chapter~11 of Grinstead and Snell~\cite{GS98}).  
A Markov chain is called absorbing if it satisfies two conditions: 
It must have at least one absorbing state~$i$, where state~$i$ is defined 
to be absorbing if and only if $\mat{P}(i,i) = 1$ and $\mat{P}(i, j) = 0$ 
for all $j \neq i$. Secondly, it must be possible to transition from every state to 
some absorbing state in a finite number of steps.
It follows directly from the construction of the graph $G'$ and the fact that the 
original graph was connected, that random walks in $G'$ define an absorbing Markov chain. 
Here, the absorbing states correspond to the set of seed nodes.

For any non-negative $r$, one can easily show that:
\begin{equation}\label{exp:rth_product_of_P}
	\mat{P}^r = \left [ \begin{array}{ll}
						\mat{Q}^r  					& \sum_{i = 0}^{r - 1}\mat{Q}^i \cdot \mat{R} \\
						 \mat{0}_{s \times (n - s)} & \mat{I}_{s \times s}
						\end{array}
				\right ].
\end{equation}  
Since we are dealing with infinite random walks, we are interested in the following 
property of absorbing Markov chains.
\begin{proposition}\label{prop:limiting_Q}
	Let $\mat{P}$ be the $n \times n$ transition matrix that defines an absorbing Markov chain
	and suppose that $\mat{P}$ is in the canonical form specified by equation~(\ref{eqn:canonical_form_P}). 
	Then
    \begin{equation}
        \lim_{r \to \infty} \mat{P}^r = \left [ \begin{array}{ll}
            \mat{0}_{(n - s) \times (n - s)} & (\mat{I} - \mat{Q})^{-1} \cdot \mat{R} \\
            \mat{0}_{s \times (n - s)}       & \mat{I}_{s \times s}
        \end{array}
        \right ].
    \end{equation}  
\end{proposition}
Intuitively, every random walk starting at a non-seed node eventually 
reaches some seed node where it is ``absorbed.'' The probability 
that such an infinite random walk starting at non-seed node~$u_i$ ends 
up at the seed node~$x_j$ is entry $(i, j)$ of the 
submatrix $\mat{X} := (\mat{I} - \mat{Q})^{-1} \cdot \mat{R}$. 

Now we can finally define the affinity vectors of non-seed nodes.
The affinity of non-seed node $u_i$ to a community~$l$ is defined as:
\begin{equation}\label{eqn:belonging_vector}
    \alpha (u_i, l) = \sum_{j = 1}^{s} \mat{X} (i, j) \cdot \alpha (x_j, l).
\end{equation}
The computational complexity of calculating these affinity values  
depends on how efficiently we can calculate the entries of $\mat{X}$, 
i.e., solve $(\mat{I} - \mat{Q})^{-1}$. 

We can reduce this problem to that of solving a system of linear equations 
of a special type which takes time $O(m \cdot \log n)$, where $m$
is the number of edges in $G$. For a proof of this see Appendix.  
 
In the next subsection, we show 
how to reduce this problem to that of solving a system of linear equations 
of a special type which takes time $O(m \cdot \log n)$, where $m$
is the number of edges in $G$.
\begin{theorem}\label{theorem:computing_NR}
Given a graph~$G$, let $\mat{P}$ be the $n \times n$ transition matrix 
defined by equation~(\ref{eqn:defining_prob}) in canonical form 
(see equation~(\ref{eqn:canonical_form_P})). Then, one can compute 
the affinity vectors of all non-seed nodes in time $O(m \cdot \log n)$ per community, 
where~$m$ is the number of edges in the graph~$G$.
\end{theorem}  



