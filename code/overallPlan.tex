\documentclass[12pt]{article}

\usepackage{verbatim}
\usepackage{amsmath}
\usepackage{times,fullpage}

\begin{document}
The program consists of the following files:

\begin{description}
	\item [\texttt{Graph}] This reads an undirected graph from file 
		in LEDA format and stores it in an adjacency list (which is 
		a vector of list of ints). Graph will have the following 
		interface: 
		\begin{enumerate}
			\item \verb|num_vertices();| 
			\item \verb|num_edges();| 
			\item \verb|Graph(const string path_to_graph_file, const string path_to_seed_node_file );|
		\end{enumerate}	
		The adjacency list also has degree information.
		\begin{enumerate}
			\item \verb|int get_degree(int);|
			\item \verb|std::vector<int> get_neighbors(int);|
			\item \verb|int get_matrix_id(int);| 
			\item \verb|display_graph();|
			\item \verb|seed_nodes(const string path_name);|
		\end{enumerate}
		The file containing the seed node information consists of 
		the seed node ids and the number of communities. The file 
		then lists the seed nodes and their affinities to the communities.
		This is stored as a vector of a vector of doubles. 
		\verb|bool is_seed( int id );|
		 
	\item [\texttt{MatrixBuilder}] The Matrix Builder builds the 
		matrices $A_1$, $D_1$, and $R$.

	\item [\texttt{matrix\_solver}] The input parameters are 
		the graph, the set of seed nodes, pointers to matrices $A_1$, $D_1$, 
		$R$, the seed node matrix. It gives back an $n \times k$ matrix of affinities.  
\end{description}

\end{document}

