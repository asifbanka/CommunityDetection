\documentclass[12pt]{article}
\usepackage{xcolor}
\usepackage{listings}
\usepackage{verbatim}
\usepackage{amsmath}
\usepackage{times,fullpage}

\lstset{language=C++,frame=none,columns=flexible}
\lstset{identifierstyle=\tt}
\lstset{commentstyle=\it}
\lstset{xleftmargin=1cm}
\lstset{ %
  backgroundcolor=\color{white},   % choose the background color; you must add \usepackage{color} or \usepackage{xcolor}
  basicstyle=\small\ttfamily,        % the size of the fonts that are used for the code
  breakatwhitespace=false,         % sets if automatic breaks should only happen at whitespace
  breaklines=true,                 % sets automatic line breaking
  captionpos=b,                    % sets the caption-position to bottom
  commentstyle=\it \ttfamily,    % comment style
  deletekeywords={...},            % if you want to delete keywords from the given language
  escapeinside={\%*}{*)},          % if you want to add LaTeX within your code
  extendedchars=true,              % lets you use non-ASCII characters; for 8-bits encodings only, does not work with UTF-8
  frame=single,                    % adds a frame around the code
  keepspaces=true,                 % keeps spaces in text, useful for keeping indentation of code (possibly needs columns=flexible)
  keywordstyle=\bf \ttfamily,       % keyword style
  language=Octave,                 % the language of the code
  morekeywords={*,...},            % if you want to add more keywords to the set
  numbers=left,                    % where to put the line-numbers; possible values are (none, left, right)
  numbersep=5pt,                   % how far the line-numbers are from the code
  numberstyle=\tiny, % the style that is used for the line-numbers
  rulecolor=\color{black},         % if not set, the frame-color may be changed on line-breaks within not-black text (e.g. comments (green here))
  showspaces=false,                % show spaces everywhere adding particular underscores; it overrides 'showstringspaces'
  showstringspaces=false,          % underline spaces within strings only
  showtabs=false,                  % show tabs within strings adding particular underscores
  stepnumber=2,                    % the step between two line-numbers. If it's 1, each line will be numbered
  stringstyle=\color{mymauve},     % string literal style
  tabsize=2,                       % sets default tabsize to 2 spaces
  title=\lstname                   % show the filename of files included with \lstinputlisting; also try caption instead of title
}


\newcommand{\program}[1]{\begin{lstlisting} \centering #1 \end{lstlisting}}
\newcommand{\code}[1]{\bf \small {\verb| #1|}}


\begin{document}
The program consists of the following files:

\begin{description}
	\item [\texttt{Graph}] This reads an undirected graph from file 
		in LEDA format and stores it in an adjacency list (which is 
		a vector of list of ints). Graph will have the following 
		interface: 
		\begin{enumerate}
			\item {\small \verb|num_vertices();|}
			\item \verb|num_edges();| 
			\item \verb|Graph(const string path_to_graph_file, const string path_to_seed_node_file );|
		\end{enumerate}	
		The adjacency list also has degree information.
		\begin{enumerate}
			\item \verb|int get_degree(int);|
			\item \verb|std::vector<int> get_neighbors(int);|
			\item \verb|int get_matrix_id(int);| 
			\item \verb|display_graph();|
			\item \verb|seed_nodes(const string path_name);|
		\end{enumerate}
		The file containing the seed node information consists of 
		the seed node ids and the number of communities. The file 
		then lists the seed nodes and their affinities to the communities.
		This is stored as a vector of a vector of doubles. 
		\verb|bool is_seed( int id );|
		 
	\item [\texttt{MatrixBuilder}] The Matrix Builder builds the 
		matrices $A_1$, $D_1$, and $R$.

	\item [\texttt{matrix\_solver}] The input parameters are 
		the graph, the set of seed nodes, pointers to matrices $A_1$, $D_1$, 
		$R$, the seed node matrix. It gives back an $n \times k$ matrix of affinities.  
\end{description}

\end{document}

