\documentclass[12pt]{article}
\usepackage{xcolor}
\usepackage{listings}
\usepackage{verbatim}
\usepackage{amsmath}

%% Fonts: Times for body text and txtt for typewriter
%\usepackage{mathptmx,fullpage}
\usepackage{lmodern,fullpage}
%\renewcommand*\ttdefault{txtt}
\usepackage[T1]{fontenc}

%% Sections headings style: sans serif
%\usepackage{sectsty}
%\allsectionsfont{\sffamily}


\lstset{language=C++,frame=none,columns=flexible}
\lstset{identifierstyle=\tt}
\lstset{commentstyle=\it}
\lstset{xleftmargin=1cm}
\lstset{ %
  backgroundcolor=\color{white},   % choose the background color; you must add \usepackage{color} or \usepackage{xcolor}
  basicstyle=\small\ttfamily,        % the size of the fonts that are used for the code
  breakatwhitespace=false,         % sets if automatic breaks should only happen at whitespace
  breaklines=true,                 % sets automatic line breaking
  captionpos=b,                    % sets the caption-position to bottom
  commentstyle=\it \ttfamily,    % comment style
  deletekeywords={...},            % if you want to delete keywords from the given language
  escapeinside={\%*}{*)},          % if you want to add LaTeX within your code
  extendedchars=true,              % lets you use non-ASCII characters; for 8-bits encodings only, does not work with UTF-8
  frame=single,                    % adds a frame around the code
  keepspaces=true,                 % keeps spaces in text, useful for keeping indentation of code (possibly needs columns=flexible)
  keywordstyle=\bf \ttfamily,       % keyword style
  language=Octave,                 % the language of the code
  morekeywords={*,...},            % if you want to add more keywords to the set
  numbers=left,                    % where to put the line-numbers; possible values are (none, left, right)
  numbersep=5pt,                   % how far the line-numbers are from the code
  numberstyle=\tiny, % the style that is used for the line-numbers
  rulecolor=\color{black},         % if not set, the frame-color may be changed on line-breaks within not-black text (e.g. comments (green here))
  showspaces=false,                % show spaces everywhere adding particular underscores; it overrides 'showstringspaces'
  showstringspaces=false,          % underline spaces within strings only
  showtabs=false,                  % show tabs within strings adding particular underscores
  stepnumber=2,                    % the step between two line-numbers. If it's 1, each line will be numbered
  stringstyle=\color{mymauve},     % string literal style
  tabsize=2,                       % sets default tabsize to 2 spaces
  title=\lstname                   % show the filename of files included with \lstinputlisting; also try caption instead of title
}

\renewcommand{\th}{\ensuremath{^{\mathrm{th}}}}
\renewcommand{\Pr}[1]{\ensuremath{\mathbf{Pr} \left [ #1 \right]}}

\title{The Overlapping Community Detection Project}

\begin{document}
\maketitle

\section{Notation} 
We use: 
\begin{enumerate}
	\item $n$ to denote the number of nodes in the network;
	\item $m$ to denote the number of edges;
	\item $s$ to denote number of seed nodes;	
	\item $k$ to denote the number of communities;
	\item $A_1$ to denote the adjacency matrix of the induced by non-seed nodes;
	\item $D_1$ to denote the diagonal matrix of the total degrees of the non-seed nodes;
	\item $x_0, \ldots, x_{s-1}$ to denote seed nodes and $u_0, \ldots, u_{n - s -1}$ 
			to denote non-seed nodes. 
\end{enumerate}

\section{Components of the program}

The program consists of the following classes.
\paragraph{The \texttt{Graph} class.} The \verb|Graph| class 
is responsible for reading an undirected graph from file 
in LEDA format which it stores in an adjacency list. 
The adjacency list itself is a \verb|std::vector| 
of \verb|std::list<int>| and this list also has the degree 
information stored explicitly. One of the constructors of this class looks like:
\begin{quote}
\verb|Graph(const char* path_to_graph_file)| 
%$~$\hspace{5mm} \verb|const std::string& path_to_seed_node_file );|
\end{quote}

The interface for this class consists of the following members: 
\begin{enumerate}
	\item \verb|num_vertices();|
	\item \verb|num_edges();| 
	\item \verb|int get_degree(int);|
	\item \verb|std::list<int> get_neighbors(int);|
	\item \verb|display_graph();|
	\end{enumerate}

\paragraph{The \texttt{SeedNode} class.} This is responsible 
for generating the set of seed nodes and their affinities 
to the respective communities. At the very least, it would read 
a file containing the seed node information which might look like 
as shown in Table~\ref{tab:seed_node}. 

\begin{table}[ht]
\centering
\begin{tabular}{llll}
\emph{num$\_$seed$\_$nodes} & \emph{num$\_$communities} &  & \\
$x_0$               & $\alpha(0, 1)$ & $\ldots$ & $\alpha(0, k)$ \\
$\vdots$ & $\vdots$ & $\vdots$ & $\vdots$ \\
$x_{s-1}$  & $\alpha(n-1, 1)$ & $\ldots$ & $\alpha(n-1, k)$ 
\end{tabular}
\caption{File format for seed node information.} \label{tab:seed_node}
\end{table}

For now, we endow it with the following member functions:
\begin{enumerate}
	\item \verb|SeedNode(const char* path_name);| \verb|//constructor|
	\item \verb|bool is_seed( int id );|
  	\item \verb|int num_seed();|
	\item \verb|int get_matrix_id(int id);| 
  	\item \verb|int num_communities();| 
  	\item \verb|int get_affinity(int seedNode,int community);| 
  	\item \verb|int get_vertex_id(int id);| to get from the matrix id to the vertex id (only needed for seed nodes)
\end{enumerate}
Later on, we might want to enrich this class by writing down how 
to generate seed nodes randomly and create their affinity vectors
using only the adjacency list and user-specified input (perhaps). 
	 
\paragraph{The \texttt{MatrixBuilder}.} The Matrix Builder builds the 
matrices $A_1$, $D_1$, and $R$ using the adjacency list and seed node 
information supplied by the \verb|Graph| class. One of the jobs that 
is has to do is create two vectors: an $(n - s)$-vector that maps 
the index of the rows of $R$ to the (non-seed) node that the index 
corresponds to; an $s$-vector that maps the column index of 
$R$ to the seed node that it corresponds to.   

\paragraph{The \texttt{matrix$\_$solver}.} This sets up the system of 
linear equations and obtains, as solutions, the $n \times k$ matrix 
of affinities. The input parameters are the graph, the set of seed nodes, 
pointers to matrices $A_1$, $D_1$, $R$, the seed node matrix. Using these, 
it creates a system of linear equations for \emph{each} community.   
For community $l$, this system looks like:
\[
	(D_1 -  A_1) \begin{pmatrix} \alpha(u_0, l) \\ \vdots \\ \alpha(u_{n - s -1}, l) \end{pmatrix}	
		= D_1 \sum_{i = 0}^{s - 1} \alpha(x_0, l) R_i. 
\] 
Here $\alpha(v, l)$ denotes the affinity of node~$v$ for community~$l$; $R_i$ 
denotes the $i\th$ column of matrix $R$. The \texttt{matrix$\_$solver} returns 
an $n \times k$ matrix of affinities in a file (whose format is described below).

\paragraph{The \texttt{CommunityClassifier}.} This reads the affinities 
from a file whose format is shown in Table~\ref{tab:node_aff}.
It then places each node into an appropriate number of communities. There 
may be multiple strategies for doing this and we might want to implement it 
so that we can call it with the strategy that we want. The output of the 
classifier is written to a file.

\begin{table}[ht]
\centering
\begin{tabular}{llll}
\emph{num$\_$vertices} & \emph{num$\_$communities} &  & \\
$v_0$ (int)              & $\alpha(0, 1)$ (double) & $\ldots$ & $\alpha(0, k)$ (double) \\
$\vdots$ & $\vdots$ & $\vdots$ & $\vdots$ \\
$v_{n-1}$  & $\alpha(n-1, 1)$ & $\ldots$ & $\alpha(n-1, k)$ 
\end{tabular}
\caption{File format for node affinities}\label{tab:node_aff}
\end{table}
%

One of the quality measures that are used is the so-called \emph{normalized
mutual information}. This is defined as follows. Let $X$ be a random variable 
taking on values $x_1, \ldots, x_r$. Then the entropy of $X$ is $H(X)$ 
which is defined as:
\[
	H(X) = \sum_{i = 1}^r p(x_i) \log_2 \frac{1}{p(x_i)}, 
\]
where $p(x_i) = \Pr{X = x_i}$.
\end{document}

